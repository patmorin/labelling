\documentclass[aspectratio=169,xcolor=dvipsnames]{beamer}
\usepackage{ods}
\usepackage{ods-figs}
\usepackage[cm]{sfmath}
%\usepackage{enumitem}

% \setlength{\leftmargini}{0pt}

\newcommand{\N}{\mathbb{N}}

\title[Adjacency Labelling]{Asymptotically Optimal Adjacency-Labelling for Planar Graphs\newline
 (and Beyond)}
\author[Dujmović, Esperet, Gavoille, Joret, Micek, Morin]{Vida Dujmović, Louis Esperet, Cyril Gavoille, Gwenaël Joret, Piotr Micek, and Pat Morin}

\begin{document}

\begin{frame}
  \titlepage
\end{frame}

\begin{frame}
  \frametitle{Adjacency Labelling}
  \framesubtitle{Labelling and Testing for a graph $G$}

  \begin{itemize}
    \item Adjacency tester $A:\{0,1\}^*\times\{0,1\}^*\to \{0,1\}$
    \item Labelling function $\ell:V(G)\to\{0,1\}^*$
    \item $(G,\ell)$ \emph{works with} $A$ if, for each $v,w\in V(G)$,
        $
            A(\ell(v),\ell(w)) = \begin{cases}
                1 & \text{if $vw\in E(G)$} \\
                0 & \text{if $vw\not\in E(G)$}
        \end{cases}
        $
    \end{itemize}
    \begin{center}
        \multiinclude[<+>][format=pdf,start=1,end=4]{figs/labelling-scheme}%
    \end{center}
\end{frame}

\begin{frame}
    \frametitle{Adjacency Labelling}
    \framesubtitle{Labelling schemes for graph families}

    \begin{itemize}
        \item<+->A graph family $\mathcal{G}$ has an $f(n)$-bit labelling scheme if \begin{itemize}
            \item there exists $A:\{0,1\}^*\times\{0,1\}^*\to\{0,1\}$
            \item for each $n$-vertex $G\in\mathcal{G}$ there exists $\ell:V(G)\to\{0,1\}^{f(n)}$ such that $(G,\ell)$ works with $A$
        \end{itemize}
    \end{itemize}
    \begin{center}
        \multiinclude[<+>][format=pdf,start=5]{figs/labelling-scheme}%
    \end{center}
\end{frame}


\begin{frame}
    \frametitle{Universal Graphs}
    \framesubtitle{Labelling schemes and universal graphs}

    \begin{itemize}
        \item<+->\textbf{Theorem (Folklore):}  If a graph family $\mathcal{G}$ has an $f(n)$-bit labelling scheme then, for each $n\in\N$ there exists a graph $U_n$ such that
        \begin{itemize}
            \item<+-> $|V(U_n)|=2^{f(n)}$
            \item<+-> For each $n$-vertex $G\in\mathcal{G}$, $G$ appears as an induced subgraph of $U_n$
        \end{itemize}
        \item<+->Proof: $V(U_n):=\{0,1\}^{f(n)}$, $E(U_n):=\{vw:A(v,w)=1\}$.
        \item<+->$U_n$ is called an \emph{(induced) universal} graph for $n$-vertex members of $\mathcal{G}$.
    \end{itemize}

\end{frame}


\begin{frame}
    \frametitle{Previous Results}
    \framesubtitle{Trees and Bounded Treewidth Graphs}

    \begin{center}
        \begin{tabular}{llc}
            \multicolumn{3}{c}{Trees} \\
            $f(n)$ & $|U_n|$ & Reference  \\ \hline
            \textcolor<2>{odsorange}{$\log n+\log\log n+O(1)$} & $O(n\log n)$ & Chung (1990)   \\
            $\log n + O(\log^* n)$ & $n2^{O(\log^* n)}$ & Alstrup-Rauhe (2006) \\
            \textcolor<3>{odsorange}{$\log n + O(1)$} & $O(n)$ & Alstrup-Dahlgaard-Knudsen (2017) \\[2em]
            \multicolumn{3}{c}{Bounded treewidth graphs} \\
            \textcolor<4>{odsorange}{$\log n + o(\log n)$} & $n^{1+o(1)}$ & Gavoille-Labourel (2007)
        \end{tabular}
    \end{center}
\end{frame}


\begin{frame}
    \frametitle{Planar Graphs}
    % \framesubtitle{Planar Graphs}
    \begin{itemize}
        \item[]
        \begin{center}
            \begin{tabular}{llcc}
                % \multicolumn{4}{c}{Planar graphs} \\
                $f(n)$ & $|U_n|$ & Reference & Techniques \\ \hline
                \textcolor<2>{odsorange}{$6\log n$} & $n^6$ & Muller (1988) & 5-degeneracy \\
                \textcolor<3>{odsorange}{$4\log n$} & $n^4$ & Kannan-Naor-Rudich (1988) & 3-orientation \\
                \textcolor<4>{odsorange}{$3\log n + O(1)$} & $O(n^3)$ & --- & arboricity + trees \\
                \textcolor<5>{odsorange}{$2\log n + o(\log n)$} & $n^{2+o(1)}$ & Gavoille-Labourel (2007) & 2-outerplanar decomp. \\
                \textcolor<6>{odsorange}{$\tfrac{4}{3}\log n + o(\log n)$} & $n^{\tfrac{4}{3}+o(1)}$ & Bonamy-Gavoille-Pilipczuk (2020) & product structure  \\
            \end{tabular}
        \end{center}
        \item<7-> \textcolor<7>{odsorange}{\textbf{Theorem 1:}} The family of planar graphs has a $(\log n + o(\log n))$-bit adjacency labelling scheme.

        \item<8-> \textbf{Corollary 1:} For each $n$, there is a graph $U_n$ with $n^{1+o(1)}$ vertices is universal for $n$-vertex planar graphs.
    \end{itemize}
\end{frame}

\begin{frame}
    \frametitle{The Strong Graph Product $\boxtimes$}

    \begin{itemize}
        \item[] For two graphs $A$ and $B$, the \emph{strong product} $A\boxtimes B$ is a graph:
        \begin{itemize}
            \item $V(A\boxtimes B):=V(A)\times V(B)$
            \item $(a_1,b_1)$ and $(a_2,b_2)$ are adjacent if and only if:
            \begin{itemize}
                \item $a_1=a_2$ and $b_1b_2\in E(B)$;
                \item $a_1a_2 \in E(A)$ and $b_1=b_2$; or
                \item  $a_1a_2 \in E(A)$ and $b_1b_2 \in E(B)$.
            \end{itemize}
        \end{itemize}
        \begin{center}
            \includegraphics{figs/product}
        \end{center}
    \end{itemize}
\end{frame}


\begin{frame}
    \frametitle{The Product Structure Theorem}

    \begin{center}
        \includegraphics{figs/product}
    \end{center}

    \begin{itemize}
        \item<+->\textbf{Product Structure Theorem (Dujmović-Joret-Micek-M-Ueckerdt-Wood 2019):} For any planar graph $G$ there exists a graph $H$ of treewidth at most $8$ and a path $P$ such that $G$ a subgraph of $H\boxtimes P$.

        \item<+->\textbf{Main Result (Here):} Let $\mathcal{G}_t$ be the family of graphs such that for each $G\in\mathcal{G}$, there exists a graph $H$ of treewidth at most $t$ and a path $P$ such that $G$ a subgraph of $H\boxtimes P$.  Then, for any fixed $t$, $\mathcal{G}$ has a $(\log n + o(\log n))$-bit labelling scheme.

        \item<+->\textbf{Applications:} bounded genus graphs, apex-minor free graphs, bounded-degree graphs from minor-closed families, $k$-planar graphs for constant $k$
    \end{itemize}
\end{frame}

\begin{frame}
    \frametitle{Warm-up: $G\subseteq P\boxtimes P$}

    \begin{center}
        \multiinclude[<+>][format=pdf,start=1]{figs/pxp}%
    \end{center}
    \begin{itemize}
        \item The label for a vertex $v=(x,i)$ has three main parts:
        \begin{itemize}
            \item A \emph{row label} of length $\log n -\textcolor{odsorange}{\log n_i} + o(\log n)$ defined by $i$
            \item A \emph{column label} of length $\textcolor{odsorange}{\log n_i} +o(\log n)$ defined by $(G_i,x)$
            \item A \emph{transition label} of length $o(\log n)$ defined by $(G_i,G_{i+1},x)$
        \end{itemize}
    \end{itemize}
\end{frame}

\begin{frame}
    \frametitle{Binary Search Trees}
    \framesubtitle{Any binary search tree can label a path}

    \begin{center}
        \multiinclude[<+>][format=pdf,start=1]{figs/bst-path}%
    \end{center}
    \begin{itemize}
        \item Label length: $\mathrm{depth}_T(v)+ O(\log\mathrm{height}(T))=\mathrm{depth}_T(v)+O(\log\log n)$ bits
    \end{itemize}
\end{frame}

\begin{frame}
    \frametitle{The Column Code Trees}
    \framesubtitle{Bulk Trees}

    \begin{center}
        \multiinclude[<+>][format=pdf,start=1,end=4]{figs/pxp-evolution}%
    \end{center}
    \begin{itemize}
        \item<4>$\sum_{i=1}^h |T_i| = O(n)$
    \end{itemize}
\end{frame}


\begin{frame}
    \frametitle{The Row Code Tree}
    \framesubtitle{Biased binary search trees}

    \begin{itemize}
        \item<+-> $\sum_{i=1}^h |T_i| = O(n)$
        \item<+-> Make a biased binary search tree $T$ with $V(T):=\{1,\ldots,h\}$  using $\mathrm{weight}(i)=|T_i|$
        \begin{itemize}
            \item<+-> Then $\mathrm{depth}_T(i) \le \log n - \textcolor{odsorange}{\log |T_i|} + O(1)$
        \end{itemize}
        \item<+-> Row $i$ label length is $\log n-\textcolor{odsorange}{\log |T_i|} + O(\log\log n)$
    \end{itemize}
\end{frame}

\begin{frame}
    \frametitle{The Column Code Tree}
    \framesubtitle{Bulk Trees}

    \begin{center}
        \multiinclude[<+>][format=pdf,start=4]{figs/pxp-evolution}%
    \end{center}
\end{frame}


\begin{frame}
    \frametitle{Bulk Insertion}
    \framesubtitle{Bulk Trees}

    \begin{center}
        \multiinclude[<+>][format=pdf,start=1]{figs/addition}%
    \end{center}
    \begin{itemize}
        \item<6> $h(T_{i+1})\le h(T_i)+c$
    \end{itemize}
\end{frame}

\begin{frame}
    \frametitle{Bulk Deletion}
    \framesubtitle{Bulk Trees}

    \begin{center}
        \multiinclude[<+>][format=pdf,start=1]{figs/deletion}%
    \end{center}
    \begin{itemize}
        \item<7-> $h(T_{i+1}) \le h(T_i)$
        \item<8-> $\log|T_i|\le \log|T_{i+1}| + c$
    \end{itemize}
\end{frame}


\begin{frame}
    \frametitle{Rebalancing}
    \framesubtitle{balance$(x,k)$)}

    Basic operation: balance$(x,k)$
    \begin{center}
        \multiinclude[<+>][format=pdf,start=1]{figs/balance}%
    \end{center}
    \begin{itemize}
        \item<5->Effect on signature can be encoded in $O(k\log\log n)$ bits
    \end{itemize}
\end{frame}

\begin{frame}
    \frametitle{Rebalancing}
    \framesubtitle{balance$(\theta)$}

    \begin{center}
        \multiinclude[<+>][format=pdf,start=1]{figs/bulkbalance}%
    \end{center}
    \begin{itemize}
        \item<11->Ensures that $h(T_i)\le \log|T_i| + O(\tfrac{1}{k}\log|T_i|)$
    \end{itemize}
\end{frame}

\begin{frame}
    \frametitle{Summary}
    \begin{itemize}[<+->]
        \item transition code length: $O(k\log\log n)$
        \item $h(T_i) \le \log|T_i| + O(\tfrac{1}{k}\log|T_i|)$
        \item label length: $h(T_i) + O(k\log\log n)\le \log|T_i| + O(\tfrac{1}{k}\log n + k\log\log n)$
        \item optimize $k$: $h(T_i) + O(\sqrt{\log n\log\log n})$
    \end{itemize}
\end{frame}

\begin{frame}
    \frametitle{Missing Pieces}

    \begin{itemize}[<+->]
        \item Subgraphs of $H\boxtimes P$ not $P\boxtimes P$
        \begin{itemize}
            \item $H$ has pathwidth $O(\log n)$
            \item $H$ is contained in an \emph{interval graph} $H^+$ of thickness $O(\log n)$
            \item Use an \emph{interval tree}
        \end{itemize}

        \item Subgraphs of $H\boxtimes P$ not \emph{induced} subgraphs
        \begin{itemize}
            \item Use $d$-orientation of $H\boxtimes P$
            \item Additional bit-vector of length $d$ indicates which induced edges are actually present
        \end{itemize}
    \end{itemize}
\end{frame}


\begin{frame}
    \frametitle{Summary and Conclusions}
    \framesubtitle{Summary}

    \begin{itemize}[<+->]
        \item Summary
        \begin{itemize}
            \item The class of $n$-vertex planar graphs has a $(\log n + O(\sqrt{\log n\log\log n}))$-bit adjacency labelling scheme.

            \item For each $n\in\N$ there exists a graph $U_n$ with $n^{1+O(\sqrt{\log n\log\log n})}$ vertices that is universal for planar graphs.
        \end{itemize}
        \item Open Problem: the lower order term
        \begin{itemize}
            \item Best upper bound: $O(\sqrt{\log n\log\log n})$ term
            \item Best lower bound: $\Omega(1)$
        \end{itemize}
    \end{itemize}
\end{frame}


\end{document}

\begin{frame}
    \frametitle{Frame Title}
    \framesubtitle{Frame Subtitle}
\end{frame}



\end{document}

\begin{frame}
  \frametitle{DLList}
  \framesubtitle{A doubly-linked list}
  \begin{block}{Theorem}
    \begin{itemize}
      \item<+->[]A DLList implements the List and Deque interfaces.
      \begin{itemize}
          \item<+->The List operations $\fn{get}(i)$, $\fn{set}(i,x)$, $\fn{add}(i,x)$ and
             $\fn{remove}(i)$ operations each run in $O(1+\min\{i,n-i\})$ time.
       \item<+->The Deque operations $\fn{addFront}(x)$, $\fn{removeFront}()$, $\fn{addBack}(x)$ and
             $\fn{removeBack}()$ operations each run in $O(1)$ time.

      \end{itemize}
    \end{itemize}
  \end{block}
\end{frame}

\closing

\end{document}
