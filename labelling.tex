\documentclass[kpfonts]{patmorin}
\usepackage{pat}
\usepackage{paralist}
\usepackage{dsfont}  % for \mathds{A}
\usepackage[utf8]{inputenc}

\usepackage{graphicx}

\setlength{\parskip}{1ex}

\DeclareMathOperator{\A}{\mathds{A}}
\DeclareMathOperator{\sn}{sn}
\DeclareMathOperator{\qn}{qn}

\renewcommand{\SS}{\mathcal{S}}


\newcommand{\aref}[1]{(X\ref{a:#1})}
\newcommand{\alabel}[1]{\label{a:#1}}

\title{\MakeUppercase{Optimal Adjacency-Labelling Schemes for Planar Graphs}}
\author{Barbados Folks}

\begin{document}
\begin{titlepage}
\maketitle

\begin{abstract}
  We describe an adjacency labelling-scheme for $n$-vertex planar graphs that assings each vertex a label of length $\log n+o(\log n)$.
\end{abstract}
\end{titlepage}
\pagenumbering{roman}
\tableofcontents

\newpage

\setcounter{page}{0}
\pagenumbering{arabic}
\section{Introduction}

We show that there exists an adjacency labelling scheme for planar graphs where each vertex of an $n$-vertex planar graph is assigned a $(\log n+o(\log n))-bit$ label.  This is optimal up to the lower-order term.


\subsection{Proof Overview}





Like Bonamy \etal\ \cite{bonamy.gavoille.ea:shorter}.  Our starting point is a recent result that characterizes planar graphs in terms of the strong product of two simpler graphs.  The \emph{strong product} $A\boxtimes B$ of two graphs $A$ and $B$ is the graph whose vertex set is the Cartesian product $V(A\boxtimes B):=V(A)\times V(B)$ and in which two vertices $v_1:=(x_1,y_1)$ and $v_2:=(x_2,y_2)$ are adjacent if and only if:
\begin{enumerate}
  \item  $v_1\neq v_2$; and
  \item $x_1=x_2$ or $x_1x_2\in E(A)$; and
  \item $y_1=y_2$ or $y_1y_2\in E(B)$.
\end{enumerate}

\begin{thm}[Dujmović \etal \cite{dujmovic.joret.ea:planar}]
  Every planar graph $G$ is the subgraph of a strong product $G^+:=H\boxtimes P$ where $H$ is a graph of treewidth at most 8 and $P$ is a path.
\end{thm}

For graph products like $G^+$, there is a natural labelling scheme: Computer a labelling scheme $\alpha:V(H)\to\{0,1\}^*$ for $H$ and a labelling scheme $\beta:V(P)\to\{0,1\}^*$ for $P$ and assign each vertex $v:=(x,y)\in V(G^*)$ the label $\mu(v):=\alpha(x),\beta(y)$.  Given two labels $\ell_1=\alpha(x_1),\beta(y_1)$ and $\ell_2=\alpha(x_2),\beta(y_2)$ for vertices $v_1=(x_1,y_1)$ and $v_2=(x_2,y_2)$ adjacency testing is done using the following formula whose three clauses follow from definition of strong product:
\[
    L(\ell_1,\ell_2):= (\ell_1\neq \ell_2) \wedge \A(\alpha(x_1),\alpha(x_2)) \wedge \A(\beta(y_1),\beta(y_2)) \enspace .
\]

\ldots



\section{Preliminaries}

For a graph $G$, we use $V(G)$ and $E(G)$ to denote the vertex and edge sets of $G$.  We use $|G|$ as a shorthand for $|V(G)|$.  

For a string $s=s_1,\ldots,s_k$, we use $|s|:=k$ to denote the length of $s$. A string $s_1,\ldots,s_k$ is \emph{prefix} of a string $t_1,\ldots,t_\ell$ if $k\le \ell$ and $s_1,\ldots,s_k=t_1,\ldots,t_k$.  A \emph{prefix-free code} $c:X\to\{0,1\}^*$ is a one-to-one function in which $c(x)$ is not a prefix of $c(y)$ for any two distinct $x,y\in X$.  Let $\N$ denote the set of non-negative integers.  The following is an old result of Elias:

\begin{lem}[Elias \cite{elias:universal}]
    There exist a prefix-free code $\gamma:\N\to\{0,1\}^*$ such that, for each $i\in\N$, $|\gamma(i)|\le \log(i+1)+O(\log\log i)$.
\end{lem}

A \emph{binary tree} $T$ is a rooted binary tree in which each node except the root is either the \emph{left} or \emph{right} child of its parent and each node has at most one left and at most one right child.  The \emph{depth} $d_T(v)$ of a node $v$ in a binary tree is the length of the unique path from the root of $T$ to $v$.  The \emph{height} of $T$ is $h(T):=\max_{v\in V(T)} d_T(v)$.

Let $T$ be a binary tree and let $v_0,\ldots,v_{r}$ be a path from the root $v_0$ of $T$ to some node $v_r$ (possibly $r=0$).  Then the \emph{signature} of $v_r$ in $T$, denoted $\sigma_T(v_r)$ is a binary string $b_1,\ldots,b_r$ where $b_i=0$ if and only if $v_{i}$ is the left child of $v_{i-1}$.  (Note that the signature of the root $v_0$ of $T$ is the empty string,  $\sigma_T(v_0)=\varepsilon$.)

A \emph{binary search tree (BST)} $T$ is a binary tree  whose node set $V(T)\subset\R$ consists of numbers and that has the \emph{BST property}:  For each node $x$ that is a left-child of its parent $p$, $x<p$ and for each node $y$ that is a right-child of its parent $p$, $p<y$.  The following observation allows us to replace (possibly large) numbers with (potentially shorter) binary strings:

\begin{obs}\obslabel{lexicographic}
  If $T$ is a binary search tree then, for any $x,y\in V(T)$, $x<y$ if and only if $\sigma_T(x)$ is lexicographically less than $\sigma_T(y)$.
\end{obs}

Let $\R^+$ denote the set of positive real numbers. The following is an easy and often-used result about biased binary search trees:

\begin{lem}\lemlabel{biased-bst}
  For any function $w:\{1,\ldots,n\}\to\R^+$, there exists a binary search tree $T$ containing $\{1,\ldots,n\}$ such that, for each $i\in\{1,\ldots,n\}$, $d_T(i)\le\log(W/w(i))$, where $W:=\sum_{i=1}^n w(i)$.
\end{lem}

To construct the tree in \lemref{biased-bst}, choose the root of $T$ to be the unique node $x\in\{1,\ldots,n\}$ such that $\sum_{z=1}^{x-1} w(i)\le W/2$ and $\sum_{z=x+1}^{n} w(i)< W/2$.  Then recurse on $\{1,\ldots,x-1\}$ and $\{x+1,\ldots,n\}$ to obtain the left and right subtrees respectively.

The following fact about binary search trees is useful, for example, in the deletion algorithms for binary search trees:

\begin{lem}\lemlabel{predecessor-encoding}
  Let $T$ be a binary search tree and let $x,y\in V(T)$ be such that $x<y$ and there is no node $z\in V(T)$ such that $x<z<y$ (i.e., $x$ and $y$ are consecutive in the sorted order of $V(T)$).  Then
  \begin{enumerate}
    \item (if $x$ has no left child) $\sigma_T(y)$ is obtained from $\sigma_T(x)$ by removing all trailing 0's and the last 1; or
    \item (if $x$ has a left child) $\sigma_T(y)$ is obtained from $\sigma_T(x)$ by appending a 0 followed by $s:=d_T(y)-d_T(x)-1$ 1's.
  \end{enumerate}
\end{lem}

Putting the a few of the preceding results together we obtain the following useful coding result:

\begin{lem}\lemlabel{row-code}
  The exists a function $A:(\{0,1\}^*)^2\to\{-1,1,\perp\}$ such that, for any $h\in\N$, and any $w:\{1,\ldots,h\}\to\R^+$ there is a prefix-free code $\alpha:\{1,\ldots,h\}\to \{0,1\}^*$ such that 
  \begin{compactenum}
    \item for each $i\in\{1,\ldots,h\}$, $|\alpha(i)|=\log W -\log w(i) + O(\log\log h)$; and
    \item for each distinct $i,j\in\{1,\ldots,h\}$, 
    \[   A(\alpha(i),\alpha(j)) 
    = \begin{cases}
       1 & \text{if $j=i+1$} \\
       -1 & \text{if $j=i-1$} \\
       \perp & \text{otherwise}
      \end{cases}
      \]
    \end{compactenum}
\end{lem}


\begin{proof}
  Define $w':\{1,\ldots,h\}\to \R^+$ as $w'(i)=w(i)+W/h$ and let $W':=\sum_{i=1}^h w'(i)=2W$.
  Using \lemref{biased-bst}, construct a biased binary search tree $T$ on $\{1,\ldots,h\}$ using $w'$ so that 
  \[   
    d_T(i)\le\log (2W)-\log(w(i)+W/h) \le \log W-\log w(i)+1 \enspace .
  \]
  Also
  \[
  d_T(i)\le\log (2W)-\log(w(i)+W/h) \le \log W-\log (W/h)+1 \le \log h + 1\enspace .
  \]
  for each $i\in\{1,\ldots,h\}$.  The code $\alpha(i$) for $i$ consists of three parts.  The first part, $\gamma(|\sigma_T(i)|)$, encodes the length of the path from the root to $i$ in $T$. The second part $\sigma_T(i)$ encodes the left/right turns along this path.  
  
  The third part $\delta(i)$ of $\alpha(i)$ is the encoding implicit in \lemref{predecessor-encoding}.  That is $\delta(i)$ consists of
  a single bit indicating whether $i$ has a left-child in $T$ and, in case $i$ does have a left-child, an Elias encoding of the value $s=d_T(i-i)-d_T(i)-1$.  More precisely, $\delta(i)=0$ or $\delta(i)=1,\gamma(s)$.  The length of $\delta(i)$ is at most $1+O(\log(s+1))=O(\log\log h)$.

  The function $A$ is now given by a simple algorithm: Given $\alpha(i)$ and $\alpha(j)$ we extract and lexicographically compare $\sigma_T(i)$ and $\sigma_T(j)$.  Assume, for now that $\sigma_T(i)$ is lexicographically less than $\sigma_T(j)$ so that, by \obsref{lexicographic}, $i < j$.  Now using $\sigma_T(j)$ and $\delta(j)$, compute $\sigma_T(j-1)$.  If $\sigma_T(j-1)=\sigma_T(i)$ then output $1$, otherwise output $\perp$.
  In the case where $\sigma_T(i)$ is lexicographically greater than $\sigma_T(j)$ we proceed in the same manner, but reversing the roles of $i$ and $j$ and outputting $-1$ in the case where $\sigma_T(i-1)=\sigma_T(j)$.
\end{proof}

\section{Subgraphs of $P\boxtimes P$}

We begin with the special case where $G\subseteq H\boxtimes P$ is an $n$-vertex graph where $H$ is a path.  For concreteness, we let $H=1,\ldots,m$ and $P=1,\ldots,h$, so that each vertex of $G$ is a pair $(x,y)\in\{1,\ldots,m\}\times \{1,\ldots,h\}$.  

For each $x\in\{1,\ldots,m\}$, we let $L_x=\{y:(x,y)\in V(H)\}$.  The idea behind our approach is to have a sequence $T_1,\ldots,T_m$ of balanced binary search trees where, for each $x\in\{1,\ldots,h\}$,
\begin{enumerate}[(PR1)]
  \item $T_x$ contains (a superset of) $\bigcup_{b\in\{-1,0,1\}} (L_{x+b}\cup \{y-1:y\in L_{x+b}\})$;
  \item The height $h(T_x)$ of $T_x$ is $\log |T_x| + o(\log n)$;
  \item $W:=\sum_{x=1}^h |T_x| = O(n)$;
  \item If $y\in V(T_x)\cap V(T_{x+1})$, then $\sigma_{T_{x+1}}(y)$ can be obtained from $\sigma_{T_{x}}(y)$ with an additional $o(\log n)$ bits that we denote by $\nu_x(y)$.  More precisely, there is a (universal) function $B:(\{0,1\}^*)^2\to\{0,1\}^*$ such that $B(\sigma_{T_{x}}(y), \nu_x(y))=\sigma_{T_{x+1}}(y)$
\end{enumerate}

Before describing the sequence of trees satisfying (PR1--4), we show how these can be used in a labelling scheme for $G$. 

Given $G$ and $T_{1},\ldots,T_m$ we use \lemref{row-code} with the weight function $w(x):=|T_x|$ to construct a code $\alpha:\{1,\ldots,m\}\to\{0,1\}^*$ where
\[  
  |\alpha(x)| = \log W-\log|T_x| + O(\log\log m) = \log n - \log|T_x| + o(\log n)
\]
for each $y\in\{1,\ldots,h\}$.  Every vertex $v=(x,y)\in V(G)$ receives a label consisting of the following:  
\begin{enumerate}
  \item $\alpha(x)$;
  \item $\gamma(|\sigma_{T_x}(y)|)$ and $\sigma_{T_x}(y)$;    
  \item $\delta_{T_x}(y)$;
  \item $\nu_x(y)$; and
  \item an array $a(v)$ of $8$ bits indicating whether each of the edges between $(x,y)$ and $(x\pm 1,y\pm 1)$ are present in $G$.  (Note that some of these 8 vertices may not even be present in $G$ in which case the resulting bit is set to 0 since the edge is not present in $G$.)
\end{enumerate}
The two major components of this label are $\alpha(x)$ and $\sigma_{T_x}(y)$ which, together have length $\log n + o(\log n)$.  The three remaining components each have size $o(\log n)$.

Given the labels of $w_1=(x_1,y_1)$ and $w_2=(x_2,y_2)$ we can test if they are adjacent as follows: Using \lemref{row-code} with $\alpha(x_1)$ and $\alpha(x_2)$, determine which of the following applies:
\begin{enumerate}
  \item $|x_1-x_2|\ge 2$: In this case we immediately conclude that $w_1$ and $w_2$ are not adjacent since $x_1x_2\not\in E(H)$.  
  
  \item $x_1=x_2$: In this case, let $x:=x_1=x_2$, let $T:=T_x$ and lexicographically compare $\sigma_T(y_1)$ and $\sigma_T(y_2)$ to determine (without loss of generality) that $y_1<y_2$.  Using $\sigma_{T}(y_2)$ and $\delta_{T}(y_2)$, compute $\delta_{T}(y_2-1)$.  If $\delta_{T}(y_2-1)\neq \delta_T(y_1)$ then immediately conclude that $w_1$ and $w_2$ are not adjacent, since $y_1y_2\not\in E(P)$.  Otherwise, we know that $y_1=y_2-1$ and $x_1=x_2$ so use the relevant bit of $a(w_1)$ (or $a(w_2)$) to determine if $w_1$ and $w_2$ are adjacent in $G$.
  
  \item $x_1=x_2-1$: In this case, use $\sigma_{T_{x_1}}(y_1)$ and $\nu_{x_1}(y_1)$ to compute $\sigma_{T_{x_2}}(y_1)$.  Now let $y:=y_2$, let $T:=T_{y}$, and proceed as in the previous case (but consulting a different bit of $a(w_1)$ in the last step.)
  
  \item $x_2=x_1-1$: In this case, use $\sigma_{T_{x_2}}(y_2)$ and $\nu_{x_2}(y_2)$ to compute $\sigma_{T_{x_1}}(y_2)$.  Now let $x:=x_1$, $T:=T_{x}$, and proceed as in the previous case (but consulting a different bit of $a(w_1)$ in the last step.)
\end{enumerate}

Thus, all that remains is to find a sequence of binary search trees $T_1,\ldots,T_m$ with properties (PR1)--(PR4). This is a data structuring problem and we use a combination of existing and new techniques from data structures to solve it.

\subsection{Fractional Cascading}

Ultimately, we want to reach a point where, for each $x\in\{1,\ldots,m-1\}$ and each $y\in V(T_x)\cap V(T_{x+1})$, the difference between $\sigma_{T_x}(y)$ and $\sigma_{T_{x+1}}(y)$ has a description $\nu_x(y)$ of length $o(\log n)$.  In order for this to be possible, we require that $V(T_x):= L_x$ not be wildly different from $V(T_{x+1}):= L_{x+1}$.  In the world of data structures, this is achieved by the technique of \emph{fractional cascading} \cite{chazelle.guibas:fractional1, chazelle.guibas:fractional2}.

For disjoint sets $X,Y\subset \R$ an integer $a$, we say that $X$ \emph{$a$-chunks} $Y$ if, for any $a$-element subset $S\subseteq Y$, there exists $x\in X$, such that $\min(S)< x<\max(S)$. Observe that, if $X$ $a$-chunks $Y$, then $|Y|\le (a-1)(|X|+1)$.

For each $x\in\{1,\ldots,m\}$, let $L^-_x:=L_x\cup \{y-1:y\in L_x\}$ and observe that $|L^-_x|\le 2|L_y|$, so $\sum_{x=1}^m|L^-_x| \le 2n$.  The following lemma is a form of fractional cascading:

\begin{lem}\lemlabel{fractional}
  There exists a universal constant $a\ge 1$ and $V_1,\ldots,V_m$ such that, 
  \begin{compactenum}
    \item for each $x\in\{1,\ldots,m\}$, $V_m\supseteq L^-_m$;
    \item for each $x\in\{1,\ldots,m-1\}$, $V_x\cap V_{x+1}$ $a$-chunks $V_{x+1}\setminus V_x$ and $V_x\setminus V_{x+1}$; and
    \item $\sum_{y=1}^h |V_y|\le bn$.
  \end{compactenum}
\end{lem}

Condition~2 in \lemref{fractional} has the following interpretation:  If we sort $V_x\cup V_{x+1}$, then we never see $a$ consecutive elements that only belong only to $V_x$ nor do we ever see $a$ consecutive elements that belong only to $V_{x+1}$.

We will use \lemref{fractional} to creating the sequence of binary search trees $T_1,\ldots,T_m$ where $V(T_x)=V_x$ for each $x\in\{1,\ldots,m\}$.  Condition~2 in \lemref{fractional} is what will ultimately make it possible to have $T_x$ and $T_{x+1}$ sufficiently similar so that, for any $y\in V_x\cap V_{x+1}$, there will be an $o(\log n)$-bit description $\nu_x(y)$ of how to modify $\sigma_{T_x}(y)$ in order to obtain $\sigma_{T_{x+1}}(y)$.

\subsection{Bulk Deletions}

\begin{lem}\lemlabel{deletion-prefix}
  For any binary search tree $T$ and any $Y\subseteq V(T)$, there exists a binary search tree $T'$ with $V(T')=V(T)\setminus Y$ such that, for every $y\in V(T')$, $\sigma_{T'}(y)$ is a prefix of $\sigma_{T}(y)$.
\end{lem}

\begin{proof}
  It suffices to prove this for a singleton set $Y$ containing one element $z$ since, for larger $Y$ we can remove the values from $T$ one at a time.  The correctness of this follows from the transitivity of the ``is a prefix of'' relation on strings.
  
  The proof is by induction on $|T|$. If $|T|=1$, then the claim is vacuous since $V(T')=\emptyset$.  Therefore assume $|T|\ge 2$.  If $z$ is leaf in $T$, then use $T'=T-\{z\}$ and the result is trivial: $\sigma_{T'}(y)=\sigma_T(y)$ for every $y\in V(T')$.  
  
  Otherwise, $z$ has at least one child.  Without loss of generality, assume $z$ has a non-empty left subtree $T_l$.  Then the largest node $z'$ in $V(T_l)$ is the largest value in $V(T)$ that is less than $z$. Inductively remove $z'$ from $T_l$ to obtain $T_l'$ and make this the left child of $z$.  Now replace $z$ with $z'$ to obtain $T'$.  
  
  That $T'$ is indeed a binary search tree is straightforward to verify.  All that remains is to prove that $\sigma_{T'}(y)$ is a prefix of $\sigma_T(y)$ for every $y\in V(T')$. There are three cases:
  \begin{enumerate}
    \item $y=z'$: Since $z$ is an ancestor of $z'$ in $T$, $\sigma_{T'}(y)=\sigma_T(z)$ is a prefix of $\sigma_T(y)=\sigma_T(z')$.  
    
    \item $y\in V(T)\setminus\{z\}\setminus V(T_l)$: In this case $\sigma_T(y)=\sigma_{T'}(y)$.  
    
    \item $y\in V(T_l')$: In this case, $\sigma_{T'}(y)=\sigma_{T}(z),0,\sigma_{T_l'}(y)$ and $\sigma_T(y)=\sigma_{T}(z),0,\sigma_{T_l}(y)$.  By induction, $\sigma_{T_l'}(y)$ is a prefix of $\sigma_{T_l}(y)$, so $\sigma_{T'}(y)$ is a prefix of $\sigma_T(y)$. \qedhere
  \end{enumerate}
\end{proof}

\lemref{deletion-prefix} is useful because it means that, for any particular node of $y$ of $T_x$, the net effect of all the deletions performed when moving from $T_x$ to $T_{x+1}$ on $\sigma(y)$ can be encoded by a single integer in the range $\{0,\ldots,h(T(x))\}$.  Using the $\gamma$-code, this requires only $O(\log h(T_x))$ bits.  If $T$ is at-all balanced, this is only $O(\log\log n)$ bits.

\subsection{Bulk Insertions}


\begin{lem}\lemlabel{chunked-addition}
  For any binary search tree $T$ and any set $Y\subset\R\setminus V(T)$ such that $V(T)$ $a$-chunks $Y$, there exists a binary search tree $T'$ with $V(T')=V(T)\cup Y$ that is a supergraph of $T$ and such that $h(T')\le h(T)+1+\log a$.
\end{lem}

\begin{proof}
  Search for each element $y\in Y$ in $T$ until the search path falls out of $T$.  This would be the natural place to add $y$ as a leaf to $T$.  This partitions $Y$ into $r:=|T|+1$ sets $Y_1,\ldots,Y_r$ and, since $V(T)$ $k$-chunks $Y$, each $Y_i$ has size at most $a-1$.  For each $Y_i$ construct a binary search tree $T_i$ of height at most $\log a$ and attach this to the appropriate location in $T$ to obtain $T'$.
\end{proof}

\lemref{chunked-addition} is useful because the set of values $Y:=L_{x+1}\setminus L_x$ that are present in $T_{x+1}$ but not in $T_x$ are $a$-chunked by $L_x\cap L_{x+1}$. Therefore, we can perform all the additions required to move from $T_x$ to $T_{x+1}$ in such a way that $\sigma_{T_x}(y)=\sigma_{T_{x+1}}(y)$ for every $y\in V_x\cap V_{x+1}$ and $h(T_{x+1})\le h(T_x)+1+\log a$.  

\subsection{Tree Splitting}

\lemref{deletion-prefix} and \lemref{chunked-addition} help control the structure of the trees $T_1,\ldots,T_m$.  However, by themselves, they are not enough to ensure that $h(T_x)=\log|T_x|+o(\log n)$, even if $T_1$ is perfectly balanced.  Indeed, \lemref{deletion-prefix} does not ensure that the height of $T$ decreases even after many deletions.  Similarly, \lemref{chunked-addition} allows the height of $T$ to increase by $1+\log a$ even in the case where only $a$ additional elements are inserted.  

In this section, we introduce what (to our knowledge) is a novel form of rebalancing that ensures that $h(T_x)=\log|T_x|+o(\log n)$.  Standard forms of rebalancing, including height-balancing, weight-balancing, and partial rebuilding seem not to be applicable in our setting because we have unusual requirements.  In particular, the requirement $h(T_x)=\log|T_x|+o(\log n)$ is quite stringent most standard binary search trees guarantee only $h(T)\le c\log |T|$ for some some constant $c>1$.  Even more exotic is that the updates come in linear-sized batches, where batch $x$ consists of the deletions $V_x\setminus V_{x+1}$ and the insertions $V_{x+1}\setminus V_x$, and we require that there exists a simple relationship between $\sigma_{T_x}(y)$ and $\sigma_{T_{x+1}}(y)$ for each $y\in V_x\cap V_{x+1}$.     

We begin with the following simple lemma:

\begin{lem}\lemlabel{split}
  Let $T$ be a binary search tree and let $y$ be any node of $T$.  Then there exists a binary search tree $T'$ with $h(T')\le h(T)+1$, $V(T')=V(T)$ whose root is $y$ and such that, for each node $z\in V(T')$, $\sigma_{T'}(z)$ is obtained by replacing a prefix of $\sigma_{T}(z)$ with a binary string from the set $\{\varepsilon\}\cup \{01^r:r\in\{0,\ldots,h(T)\}\cup \{10^r:r\in\{0,\ldots,h(T)\}$.  
\end{lem}

\begin{proof}
Proof by (crappy) illustration:
  \begin{center}
    \includegraphics[width=.6\textwidth]{images/split}
  \end{center}
\end{proof}

\lemref{split} allows us to perform a perfect split at the root of $T$ and still maintain an $O(\log h(T))$-length description of the changes required to obtain $\sigma_{T'}(z)$ from $\sigma_T(z)$.  A useful fact about \lemref{split} is that the left and right subtree of $T'$ each have height at most $h(T)$.
The following lemma uses \lemref{split} recursively:

\begin{lem}\lemlabel{mini-split}
  Let $T$ be a binary search tree and let $y_1<\cdots<y_c$ be a set of nodes in $T$. Then there exists a binary search tree $T'$ with $V(T')=V(T)$, $h(T')\le h(T)+\log c$, and in which $T'[\{y_1,\ldots,y_c\}]$ is a tree of height at most $\log c$ that contains the root of $T'$ and such that, for each node $z\in V(T')$, there is a binary string of length $O((1+\log c)\log h(T))$ that describes how to convert $\sigma_T(z)$ into $\sigma_{T'}(z)$.  Furthermore, each tree in the forest $T'-\{y_1,\ldots,y_c\}$ is a tree of height at most $h(T)$.
\end{lem}

\begin{proof}
  The proof is by induction on $c$.  If $c=0$, the statement is trivially true with $T':=T$.  If $c\ge 1$,  apply \lemref{split} to $y_{\lceil c/2\rceil}$ to obtain a tree $T_\times$ rooted at at $y_{\lceil c/2\rceil}$.  Next apply induction on the left subtree of $T_\times$ with the set $y_1,\ldots,y_{\lceil c/2\rceil -1}$ and apply induction on the right subtree of $T_\times$ with the set $y_{\lceil c/2\rceil +1},\ldots,y_c$.  Let $T'$ be the tree whose root is $y_{\lceil c/2\rceil}$ and whose left and right subtrees are the result of these two applications of induction.
  
  It is straitforward to check that $h(T')\le h(T)+\log c)$ and that $T[\{y_1,\ldots,y_c\}]$ is a tree of height at most $\log c$ that contains the root of $T'$.  The condition involving $\sigma_T(y)$ and $\sigma_{T'}(y)$ follows from the fact that any node $y\in V(T)$ is involved in at most $\log c$ inductive calls and each such call involves one application of \lemref{split}.  The effect of each application of \lemref{split} on $\sigma(y)$ can be recorded by $O(\log h(T))$ bits for a total of $O((1+\log c)\log h(T))$ bits.  Finally, the condition involving the heights of trees in $T'-\{y_1,\ldots,y_c\}$ follows from the fact that the depth any node $z$ only increases (by 1) for each element of $y_1,\ldots,y_c$ on the path from the root of $T'$ to $y$.
\end{proof}

\begin{lem}\lemlabel{multi-split}
  Let $k,c\ge 1$ be integers, and let $T$ be a binary search tree.  Then there is a binary search tree $T'$ with $V(T')=V(T)$ in which
  \begin{compactenum}
    \item $h(T')\le h(T)+1$; 

    \item  each of the subtrees $T_i'$, $i\in\{1,\ldots,m'\}$ rooted at a depth-$(k+1)$ node of $T'$ has size at most $|T|/2^k$; and
    
    \item for each $y\in V(T)$, there exists an $O(k\log h(T))$-bit description of the changes required to create $\sigma_{T'}(y)$ from $\sigma_T(y)$.
  \end{compactenum}
\end{lem}

\begin{proof}
  Let $Z\subset V(T)$ be the set of at most $2^k-1$ nodes of $T$ of depth less than $k$.  Then $T-Z$ is a forest consisting of $m\le 2^{k}$ trees $T_1,\ldots,T_m$.  Each such tree $T_i$ has height at most $h(T)-k$.
  
  Select the nodes $Y:=\{y_1,\ldots,y_{2^k}-1\}$ of $T$ where each $y_j$ has rank $\lfloor j|T|/2^k\rfloor$ in the set $V(T)$.\footnote{For a finite $X\subset\R$, the \emph{rank} of $x\in S$ is $|\{x'\in S: x'<x\}$.}  For each $j\in\{1,\ldots,m\}$, apply \lemref{mini-split} to the subtree $T_j$ with the at most $2^k$ values $Y\cap V(T_j)$ to obtain a new subtree $T_{j}'$.  Let $T^@$ be the tree obtained by replacing each of $T_1,\ldots,T_m$ with $T_1',\ldots,T_m'$, respectively.
  
  Now, $Z\cup Y$ has size at most $2(2^{k}-1)< 2^k-1$ and $T^@[Z\cup Y]$ is a connected subtree that contains the root of $T^@$.  Construct the tree $T'$ by building a complete binary search tree $T'_0$ of height at most $k$ with $V(T_0')=Z\cup Y$ and then attaching each tree in the forest $T^@-(Z\cup Y)$ to the appropriate location in $T'_0$.  Since $T'_0$ has height at most $k$ and each tree in this forest has height at most $h(T)-k$, $h(T')\le h(T)+1$.
  
  For the final condition, observe that, for each each node $y\in (Z\cup Y)\cap V(T_i)$ we can obtain $\sigma_{T^@}(y)$ from an additional $O(k\log h(T))$ bits as described in \lemref{mini-split}. From $\sigma_{T^@}(y)$ we can obtain $\sigma_{T'}(y)$ be deleting a prefix whose length can be specified using $O(\log h(T))$ bits and replace this with a sequence of at most $k+1$ bits describing a path from the root of $T'_0$ to an external node.
\end{proof}

\subsection{Rebalancing}

We can now describe the sequence of binary search trees $T_1,\ldots,T_m$ satisfying (PR1)--(PR4) required by the labelling scheme.  The easiest way to describe these trees is in terms of a tree $T$ that supports two operations:  bulk-deletion in which a set $D$ of at most $|T|(1-1/a)$ values are removed from $T$ and bulk-insertion in which a set $I$ of at most $a|T|$ new values are added into $T$.  In the case of bulk-insertion, the set $D$ of newly added values are $a$-chunked by $V(T)$.

Each bulk-deletion does not increase $h(T)$ and decreases $|T|$ be a factor of at most $a$. So, if $T^0_1$ denotes the tree $T$ before the bulk-deletion, $h(T^0_1)=\log |T_1|+r$, and $T_1'$ denotes the tree $T$ after the bulk-deletion, we have
\[  h(T_1') \le h(T^0_1) = \log|T^0_1|+r \le \log(a|T_1'|) + r \le \log|T_1'| + r+\log a \enspace . \]
Similarly, if $T_2^0$ denotes the tree $T$ after a the bulk-insertion into $T_1'$, we have
\[
   h(T_2^0) \le h(T_1')+\log a \le \log|T_1'|+r+2\log a \le \log|T_2^0|+r+2\log a \enspace .
\]
Thus, a single round of bulk-insertion/bulk-deletion does not increase the height of $T$ by more than a constant more than its current height.  The trick is to devise a rebalancing scheme that only allows this to continue for $o(\log n)$ rounds before guaranteeing that the tree returns to a more balanced state.

To achieve this we use \lemref{multi-split} in a round-robin fashion.  
Observe that, for each $i$, we have the inequality 
\[   |T_i|\le a^{i-1}|T_1|\le a|T_1| \enspace . \]

Before performing the bulk-insertion/deletion operations that take $T_1$ onto $T_2$, we first apply \lemref{multi-split} on $T_1$ with a value $k\in o(\log n)$ to be discussed shortly to yield the tree $T^0_1$ on which the bulk-deletion/insertion operations are performed.  This ensures that each depth-$(k+1)$ subtree of $T_1$ has size at most $|T_1|/2^k$.  Since the inserted elements are $a$-chunked by $V(T_1)$, this also ensures that each depth-$(k+1)$ subtree of $T_2$ has size at most $(a-1)(|T_1|/2^k+1)\le 2a|T_1|/2^k$

In general, we ensure that each subtree of $T_i$ rooted at a depth $(i-1)(k+1)$ node of $T_i^0$ has size at most 
\begin{equation}     \frac{(2a)^{i-1}|T_1|}{2^{(i-1)k}} =   |T_1|\left(\frac{2a}{2^k}\right)^{i-1}  \eqlabel{invariant}
\end{equation}
If $T_i$ has this property, then applying \lemref{multi-split} to each of the depth-$(i-1)(k+1)$ nodes ensures that each subtree of $T_{i}^0$ rooted at a depth-$i(k+1)$ has size at most 
\[
   |T_1|\left(\frac{2a}{2^k}\right)^{i-1}/2^k  \enspace .
\]
The bulk-deletions performed to create $T_i'$ do not increase the size of any subtrees and the bulk-insertions performed into $T_i'$ to create $T_{i+1}$ increase the size of each subtree by a factor of at most $2a$.  Therefore, each subtree of $T_{i+1}$ rooted at a depth-$i(k+1)$ node has size at most
\[
  |T_1|\left(\frac{2a}{2^k}\right)^{i}  \enspace ,
\]
thus guaranteeing that \eqref{invariant} holds for $T_{i+1}$.

Note that we can stop when $(2^k/2a)^{i-1}\ge |T_1|$, which occurs when
\[
    i \ge \frac{\log|T_1|}{k-1-\log a} + 1 \enspace .
\]

Continue rewrite from here....  
\hrule

 

The idea is to rebalance $T$ using \lemref{multi-split} after each set of


















The tree $T_1$ is a perfectly balanced binary search tree on node set $V_1$.  To obtain $T_2$, we first use \lemref{chunked-addition} to insert the elements of $V_2\setminus V_1$ into $T_1$ to obtain a binary search tree tree $T_1'$ with node set $V_1\cup V_2$.   Note that, since $V_1\cap V_2$ $a$-chunks $V_2$, these insertions do not add more than $(a-1)\cdot(|T_1|+1)\le 2(a-1)|T_1|$ nodes. Therefore $|T_1'|\le 2a|T_1|$.  Also $h(T_1')\le h(T_1)+\log a$.  

If $|T_1'|<2^k$ (for some $k\in o(\log n)$ to be specified later), then life is easy.  We make $T_2$ a perfectly balanced binary search tree and, for each node $x$ we store $\sigma_{T_1}(x)$ and $\sigma_{T_2}(x)$ explicitly.  In this case $\nu_1(x)$ consists of a few code bits, $\gamma(|\sigma_{T_1}(x)|)$, $\sigma_{T_1}(x)$, $\gamma(|\sigma_{T_2}(x)|)$, and $\sigma_{T_2}(x)$.  In this case we then proceed the same way beginning from $T_2$.

On the other hand, if $|T_1'|\ge 2^k$ , then we apply \lemref{multi-split} to $T_1'$ to obtain a new tree $T_1''$ of height at most $h(T_1')+O(\log a)$.  Consider the $2^{k+1}$ subtrees of $T_1''$ rooted at the depth-$k+1$ nodes of $T_1''$.  Each of these subtrees has size at most $|T_1''|/2^k\le 2a|T_1|/2^k$.  Next we apply use \lemref{deletion-prefix} to delete elements of $V_1\setminus V_2$ and we define $T_2$ to be the resulting tree.  Deletions do not increase the depth of any node, so $h(T_2)\le h(T_1'')\le h(T_1')+1\le h(T_1)+1+\log a$.  For each node $x\in T$, we make $\nu_1(x)$ by using the code described in \lemref{multi-split} followed by the code described in \lemref{deletion-prefix}.

Next, to obtain $T_3$ from $T_2$ we first use \lemref{chunked-addition} to insert the elements of $V_3\setminus V_2$ into $T_2$ to obtain a binary search tree $T_2'$ with node set $V_2\cup V_3$.   Again, these insertions do not add more than $2a|T_2|$ nodes and $h(T_2')\le h(T_2)+\log a\le h(T_1)+1+2\log a$.  Now, $T_2'$ has at most $2^{k+1}$ subtrees whose roots have depth $k+1$ and each of these subtrees has size at most $(2a)^2|T_1|/2^k$.  If $(2a)^2|T_1|/2^k < 2^k$ then we rebuild each of these subtrees into a perfectly balanced binary search tree of height at most $k-1$.  Everything is easy and we can get away with $\nu_2(x)$ of size $O(k)$ and codes of length $O(k)\in o(\log n)$.  At this point we proceed from the beginning again, as if $T_3$ were $T_1$.

If $(2a)^2|T_1|/2^k > 2^k$, then we apply \lemref{multi-split} to each of the at most $2^{k+1}$ depth-$k+1$ subtrees of $T_2'$ to obtain a new tree $T_2''$.  The tree $T_2''$ has at most $2^{2(k+1)}$ subtrees rooted at depth-$2(k+1)$ nodes that each have size at most $a^2|V_1|/2^{2k}$. Now, $h(T_2'')\le h(T_2)+O(\log a)$. Again, we apply \lemref{deletion-prefix} to delete elements of $V_2\setminus V_3$ and we define $T_3$ to be the resulting tree.  The signatures of nodes in $T_3$ have length $h(T_1)+O(k+\log a)$.  Furthermore, for each node $x\in V(T_2'')$ there is only one application of \lemref{multi-split} on a subtree of $T_2'$ that contains $x$, so $\nu_2(x)$ has length $O(k+\log a\log h(T_2'))$.

After $r$ iterations of this kind, the subtrees rooted at nodes of depth $r(k+1)$ have size at most
\[
     f(r) := 2a\left(\frac{2a}{2^k}\right)^r|V_1|
\]
and have height at most $h(T_1)-r(k+1)+r(1+\log a)$.  Therefore $h(T_r)\le h(T_1)+r(1+\log a)$.

We quit if $f(r)<2^k$, which occurs when
\[
   r > \frac{k + \log |T_1| + \log(2a)}{(k-\log 2a)} \enspace .
\]
Observe that 
\[
   h(T_r) \le h(T_1)+r(1+\log a) \le h(T_1) + O(h(T_1)/k) = (1+O(1/k))\log|T_1|
\]
On the other hand, $|T_r|\ge |T_1|/a^r$, so
\[  \log |T_r| \ge \log|T_1| - r\log a \ge (1-O(1/k))\log|T_1| \]
Putting these things together, we see that 
\[
   h(T_r) \le (1+O(1/k))\log |T_r| \enspace .
\]
Therefore, for all $r$, $|\sigma_{T_r}(x)|\le (1+O(1/k))\log|T_r|$.  The size of $\nu_r(x)$ is $O(k+\log(h(T_r))$.  If we take $k=\sqrt{\log n}$, for example, then $h(T_r)=\log |T_r|+O(\sqrt{\log n})$ and $|\nu_r(x)|=O(\sqrt{\log n})$.  Thus we get a sequence of trees $T_1,\ldots,T_h$ that satisfy (PR2) with $h(T_y)=\log_|T_y| + O(\sqrt{\log n})$ and (PR4) with $|\nu_y(x)|=O(\sqrt{\log n})$.

In this way, we obtain an adjacency labelling scheme for any $n$-vertex graph $G\subseteq P\times P$.

\section{So What?}

The obvious question is: Why do we care about labelling schemes for $n$-vertex subgraphs of $P\boxtimes P$? The answer is that we probably don't.  However, this idea could potentially extend to $n$-vertex subgraphs of $H\boxtimes P$ where $H$ has constant treewidth.  Indeed, it's not hard to imagine a data structure for a fixed $H$, that maintains a vertex-induced subgraph of $H$ and supports bulk insertions and deletions (subject to restrictions like those above) and has properties analagous to (PR1)--(PR4).

Also, even if this doesn't extend to the case where $G$ is an $n$-vertex subgraph of $H\boxtimes P$ for some $H$ of constant treewidth, David Wood makes the point that $n$-vertex subgraphs of $P\boxtimes P\boxtimes\cdots\boxtimes P$ are also interesting, because of \url{https://link.springer.com/article/10.1007/s00493-007-2183-y}.




\bibliographystyle{plain}
\bibliography{labelling}

\end{document}
