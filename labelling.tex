\documentclass[kpfonts]{patmorin}
\usepackage{pat}
\usepackage{paralist}
\usepackage{dsfont}  % for \mathds{A}
\usepackage[utf8]{inputenc}

\usepackage{graphicx}

\newcommand{\snote}[1]{\fcolorbox{red}{yellow}{#1}}
\newcommand{\pnote}[1]{\ \newline\noindent\fcolorbox{red}{yellow}{\begin{minipage}{\textwidth}#1\end{minipage}}}
\setlength{\parskip}{1ex}

\DeclareMathOperator{\A}{\mathds{A}}
\DeclareMathOperator{\sn}{sn}
\DeclareMathOperator{\qn}{qn}

\renewcommand{\SS}{\mathcal{S}}


\newcommand{\aref}[1]{(X\ref{a:#1})}
\newcommand{\alabel}[1]{\label{a:#1}}

\title{\MakeUppercase{Optimal Adjacency-Labelling Schemes for Planar Graphs}}
\author{Barbados Folks}

\begin{document}
\begin{titlepage}
\maketitle

\begin{abstract}
  We show that there exists an adjacency labelling scheme for planar graphs where each vertex of an $n$-vertex planar graph $G$ is assigned a $(\log n+O(\sqrt{\lg n\lg\lg n}))$-bit label and the labels of two vertices $u$ and $v$ are sufficient to determine if $uv$ is an edge of $G$.  This is optimal up to the lower order term and is the first such optimal result.  An alternative, but equivalent, interpretation of this result is that, for every $n$, there exists a graph $U_n$ with $n2^{O(\sqrt{\lg n\lg\lg n})}$ vertices and edges such that every $n$-vertex planar graph is an induced subgraph of $U$.  Again, this is the first near-linear upper bound on the size of $U_n$.
\end{abstract}
\end{titlepage}
\pagenumbering{roman}
\tableofcontents

\newpage

\setcounter{page}{0}
\pagenumbering{arabic}
\section{Introduction}

In this paper, which is about binary encodings, $\log x:=\log_2 x$ denotes the binary logarithm of $x$ and, for convenience, $\lg x = \log\max\{1,x\}$.  We show that there exists an \emph{adjacency labelling scheme} for planar graphs where each vertex of an $n$-vertex planar graph $G$ is assigned a $(\log n+O(\sqrt{\lg n\lg\lg n}))$-bit label and the labels of two vertices $u$ and $v$ are sufficient to determine if $uv$ is an edge of $G$.  This is optimal up to the lower order term.  An alternative, but equivalent, interpretation of this result is that, for every integer $n\ge 1$, there exists a graph $U_n$ with $n2^{O(\sqrt{\lg n\lg\lg n})}$ vertices such that every $n$-vertex planar graph is an induced subgraph of $U_n$.  (The connection between labelling schemes and universal graphs is discussed, for example, in Spinrad's monograph \cite[Section~2.1]{spinrad:efficient}.) 

This is the latest in a series of results dating back to Kannan, Naor, and Rudich \cite{kannan.naor.ea:implicit0,kannan.naor.ea:implicit} and Muller \cite{muller:local} who describe adjacency labelling schemes for planar using labels of length $O(\log n)$.  Muller's scheme, based on the fact that planar graphs are 5-degenerate uses labels of length $6\lceil\log n\rceil$.  Kannan, Naor, and Rudich use the fact that planar graphs have arboricity 3 (so their edges can be partitioned into three forests \cite{nash-williams:xx}) to devise an adjacency labelling scheme for planar graphs whose labels have length at most $4\lceil\log n\rceil$.  

A number of adjacency-labelling schemes for $n$-vertex forests that use labels of length $(1+o(1))\log n$ have been devised \cite{chu, ar02, adbk17}.  Combined with the fact that planar graphs have arboricity 3, these schemes imply adjacency labelling schemes for $n$-vertex planar graphs whose labels have length $(3+o(1))\log n$.   (Schnyder's representation of planar graphs in terms of dimension-3 posets \cite{schnyder:planar} also implies a $3\lceil\log n\rceil$-bit labelling scheme for planar graphs.)

A further improvement, also based on the idea of partitioning the edges of a planar graph into simpler graphs was obtained by Gavoille and Labourel \cite{gavoille.labourel:smaller}.  They describe an adjacency labelling scheme for $n$-vertex graphs of bounded treewidth that use labels of length $(1+o(1))\log n$.  The edges of a planar graph can be partitioned into two sets, each of which induces a bounded treewidth graph.  This results in a $(2+o(1))\log n$-bit labelling scheme for $n$-vertex planar graphs.

Very recently, Bonamy, Gavoille, and Pilipczuk \cite{bonamy.gavoille.ea:shorter} described a $(\tfrac43+o(1))\log n$-bit labelling scheme for planar graphs based on a recent \emph{graph product structure theorem} of Dujmović \etal\ \cite{dujmovic.joret.ea:planar}.  This product structure theorem states that any planar graph is a subgraph of the strong product $H\boxtimes P$ where $H$ is a bounded-treewidth graph and $P$ is a path. See \figref{product}. It is helpful to think of $H\boxtimes P$ as a graph whose vertices can be partitioned into $h:=|V(P)|$ \emph{rows} $H_1,\ldots,H_{h}$, each of which induces a copy of $H$ and with vertical and diagonal edges joining corresponding and adjacent vertices between consecutive rows.  We now quickly sketch the construction of Bonamy, Gavoille, and Pilipczuk \cite{bonamy.gavoille.ea:shorter}.

\begin{figure}
  \begin{center}
    \includegraphics{figs/product}
  \end{center}
  \caption{The strong product $H\boxtimes P$ of a binary tree $H$ and a path $P$.}
\end{figure}  
% In the following two paragraphs we omit $o(\log n)$ terms.

The product structure theorem quickly leads to a $(1+o(1))\log(mh)$-bit labelling scheme where $m:=|V(H)|$ and $y:=|V(P)|$ by using a $(1+o(1))\log m$-bit labelling scheme for $H$ (a bounded treewidth graph) and a $(1+o(1))\log h$-labelling scheme for $P$ (a path).  In a nutshell, Bonamy, Gavoille, and Pilipczuk's improve upon this by cutting $P$ (and hence $G$) into subpaths of length $n^{\tfrac13}$ by removing $O(n^{2/3})$ vertices of $G$ that have a neighbourhood of size $O(n^{2/3})$. The resulting (cut) graph is a subgraph of $H\boxtimes P'$ where $P'$ is a path of length $O(n^{1/3})$ so it has a labelling scheme in which each vertex has a label of length $(1+o(1))\log (xn^{\tfrac13})\le (\tfrac43+o(1))\log n$.  A slight modification of this scheme allows for the $O(n^{\tfrac23})$ vertices adjacent to the cuts to have shorter labels, of length only $(\tfrac23+o(1))\log n$.  The cut vertices and the vertices adjacent to them induce a bounded-treewidth graph of size $O(n^{\tfrac23})$.  The vertices in this graph receive secondary labels of length $(\tfrac23+o(1))\log n$.  In this way, every vertex receives a label of length at most $(4/3)\log n$.

The adjacency labelling scheme described in the current paper is also based on the product structure theorem for planar graphs, but it avoids cutting the path $P$, and thus avoids boundary vertices that take part in two different labelling schemes.  Instead, it uses a weighted labelling scheme for $P$ in which row $i$ receives a label of length $\log n-\log W_i$ where $W_i$ is related to the number of vertices of $G$ contained in $H_i$ and $H_{i-1}$.  The vertices of $G$ in row $i$ participate in a secondary labelling scheme for the subgraph of $G$ contained in $H_i$ and $H_{i-1}$ and the labels in this scheme have length $\log W_i$. Thus every vertex receives two labels, one of length $\log n-\log W_i$ and another of length $\log W_i$ for a total label length of $\log n$.  

The key new technique that allows all of this to work is that the labelling schemes of the rows are not independent.  All of these labelling schemes are based on a single balanced binary search tree $T$ that undergoes insertions and deletions resulting in a sequence of related binary search trees $T_1,\ldots,T_h$ where each $T_i$ contains all vertices of $G$ in $H_{i}$ and $H_{i-1}$ and the label assigned to a vertex of $H_i$ is (essentially) based on a path from the root of $T_i$ to some vertex of $T_i$.  By carefully designing these trees, the labels for $v$ in $H_i$ can be obtained, with $o(\log n)$ additional bits from the label for $v$ in $H_{i-1}$.  


% \subsection{Proof Overview}
% 
% Like Bonamy \etal\ \cite{bonamy.gavoille.ea:shorter}.  Our starting point is a recent result that characterizes planar graphs in terms of the strong product of two simpler graphs.  The \emph{strong product} $A\boxtimes B$ of two graphs $A$ and $B$ is the graph whose vertex set is the Cartesian product $V(A\boxtimes B):=V(A)\times V(B)$ and in which two vertices $v_1:=(x_1,y_1)$ and $v_2:=(x_2,y_2)$ are adjacent if and only if:
% \begin{enumerate}
%   \item  $v_1\neq v_2$; and
%   \item $x_1=x_2$ or $x_1x_2\in E(A)$; and
%   \item $y_1=y_2$ or $y_1y_2\in E(B)$.
% \end{enumerate}
% 
% \begin{thm}[Dujmović \etal \cite{dujmovic.joret.ea:planar}]
%   Every planar graph $G$ is the subgraph of a strong product $G^+:=H\boxtimes P$ where $H$ is a graph of treewidth at most 8 and $P$ is a path.
% \end{thm}
% 
% \ldots

\snote{TODO: Gwen and Piotr's comments.}

% For graph products like $G^+$, there is a natural labelling scheme: Computer a labelling scheme $\alpha:V(H)\to\{0,1\}^*$ for $H$ and a labelling scheme $\beta:V(P)\to\{0,1\}^*$ for $P$ and assign each vertex $v:=(x,y)\in V(G^*)$ the label $\mu(v):=\alpha(x),\beta(y)$.  Given two labels $\ell_1=\alpha(x_1),\beta(y_1)$ and $\ell_2=\alpha(x_2),\beta(y_2)$ for vertices $v_1=(x_1,y_1)$ and $v_2=(x_2,y_2)$ adjacency testing is done using the following formula whose three clauses follow from definition of strong product:
% \[
%     L(\ell_1,\ell_2):= (\ell_1\neq \ell_2) \wedge \A(\alpha(x_1),\alpha(x_2)) \wedge \A(\beta(y_1),\beta(y_2)) \enspace .
% \]
% 
% \ldots



\section{Preliminaries}

For a graph $G$, we use $V(G)$ and $E(G)$ to denote the vertex and edge sets of $G$.  We use $|G|$ as a shorthand for $|V(G)|$. For a vertex $v\in V(G)$, let $N_G(v):=\{w\in V(G): vw\in E(G)\}$ and $B_G(v):=N_G(v)\cup\{v\}$ denote the open neighbourhood and closed neighbourhood of $v$ in $G$, respectively.

For a string $s=s_1,\ldots,s_k$, we use $|s|:=k$ to denote the length of $s$. A string $s_1,\ldots,s_k$ is \emph{prefix} of a string $t_1,\ldots,t_\ell$ if $k\le \ell$ and $s_1,\ldots,s_k=t_1,\ldots,t_k$.  A \emph{prefix-free code} $c:X\to\{0,1\}^*$ is a one-to-one function in which $c(x)$ is not a prefix of $c(y)$ for any two distinct $x,y\in X$.  Let $\N$ denote the set of non-negative integers.  The following is an old result of Elias:

\begin{lem}[Elias \cite{elias:universal}]\lemlabel{elias}
    There exist a prefix-free code $\gamma:\N\to\{0,1\}^*$ such that, for each $i\in\N$, $|\gamma(i)|\le 2\lfloor\log(i+1)\rfloor + 1\in O(\log(i+1))$.
\end{lem}

A \emph{binary tree} $T$ is a rooted binary tree in which each node except the root is either the \emph{left} or \emph{right} child of its parent and each node has at most one left and at most one right child.  For any node $v$ in $T$, $P_T(v)$ denotes the path from the root of $T$ to $v$.  The \emph{length} of a path $P$ is the number of edges in $P$.  The \emph{depth}, $d_T(v)$ of $v$ is the length of $P_T(v)$.  The \emph{height} of $T$ is $h(T):=\max_{v\in V(T)} d_T(v)$.  A \emph{perfectly balanced} binary tree is any binary tree $T$ with $h(T)=\lfloor\log|T|\rfloor$.

A binary tree is \emph{full} if each non-leaf node has exactly two children. For a binary tree $T$, we let $T^+$ denote the full binary tree obtained by attaching $2-c$ leaves to each node of $T$ with $c$ children.  We call the leaves in $V(T^+)\setminus V(T)$ the \emph{external nodes} of $T$.

A node $a\in V(T)$ is an \emph{ancestor} of $v\in V(T)$ if the path in $T$ from the root of $T$ to $v$ includes $a$.  (Note that $v$ is an ancestor of itself.) For a node subset $X\subseteq V(T)$, the \emph{lowest common ancestor} of $X$ is the maximum-depth node $a\in V(T)$ such that $a$ is an ancestor of $v$ for each $v\in X$.

Let $v_0,\ldots,v_{r}$ be a path from the root $v_0$ of $T$ to some node $v_r$ (possibly $r=0$).  Then the \emph{signature} of $v_r$ in $T$, denoted $\sigma_T(v_r)$ is a binary string $b_1,\ldots,b_r$ where $b_i=0$ if and only if $v_{i}$ is the left child of $v_{i-1}$.  (Note that the signature of the root $v_0$ of $T$ is the empty string,  $\sigma_T(v_0)=\varepsilon$.)

A \emph{binary search tree (BST)} $T$ is a binary tree  whose node set $V(T)\subset\R$ consists of real numbers and that has the \emph{BST property}:  For each node $x$, $z<x$ for each node $z$ in $x$'s left subtrees and $z>x$ for each node $z$ in $x$'s right subtree. For any $x\in\R\setminus V(T)$, the \emph{search path} $P_T(x)$ in $T$ is the unique root-to-leaf path $v_0,\ldots,v_r$ in $T^+$ such that adding $x$ as a (left or right, as appropriate) leaf child of $v_{r-1}$ in $T$ would result in a binary search tree $T'$ with $V(T')=V(T)\cup\{x\}$.

The following observation allows us to replace (possibly large) numbers with (potentially shorter) binary strings:

\begin{obs}\obslabel{lexicographic}
  If $T$ is a binary search tree then, for any $x,y\in V(T)$, $x<y$ if and only if $\sigma_T(x)$ is lexicographically less than $\sigma_T(y)$.
\end{obs}

Let $\R^+$ denote the set of positive real numbers. The following is an easy and often-used result about biased binary search trees:

\begin{lem}\lemlabel{biased-bst}
  For any function $w:\{1,\ldots,n\}\to\R^+$, there exists a binary search tree $T$ containing $\{1,\ldots,n\}$ such that, for each $i\in\{1,\ldots,n\}$, $d_T(i)\le\log(W/w(i))$, where $W:=\sum_{i=1}^n w(i)$.
\end{lem}

To construct the tree in \lemref{biased-bst}, choose the root of $T$ to be the unique node $x\in\{1,\ldots,n\}$ such that $\sum_{z=1}^{x-1} w(i)\le W/2$ and $\sum_{z=x+1}^{n} w(i)< W/2$.  Then recurse on $\{1,\ldots,x-1\}$ and $\{x+1,\ldots,n\}$ to obtain the left and right subtrees, respectively.

The following fact about binary search trees is useful, for example, in the deletion algorithms for several types of balanced binary search trees \cite[Section~6.2.3]{morin:open}:

\begin{lem}\lemlabel{predecessor-encoding}
  Let $T$ be a binary search tree and let $x,y\in V(T)$ be such that $x<y$ and there is no node $z\in V(T)$ such that $x<z<y$ (i.e., $x$ and $y$ are consecutive in the sorted order of $V(T)$).  Then
  \begin{enumerate}
    \item (if $y$ has no left child) $\sigma_T(x)$ is obtained from $\sigma_T(y)$ by removing all trailing 0's and the last 1; or
    \item (if $y$ has a left child) $\sigma_T(x)$ is obtained from $\sigma_T(y)$ by appending a 0 followed by $s:=d_T(y)-d_T(x)-1$ 1's.
  \end{enumerate}
\end{lem}

Putting some of the preceding results together we obtain the following useful coding result:

\begin{lem}\lemlabel{row-code}
  The exists a function $A:(\{0,1\}^*)^2\to\{-1,1,\perp\}$ such that, for any $h\in\N$, and any $w:\{1,\ldots,h\}\to\R^+$ there is a prefix-free code $\alpha:\{1,\ldots,h\}\to \{0,1\}^*$ such that 
  \begin{compactenum}
    \item for each $i\in\{1,\ldots,h\}$, $|\alpha(i)|=\log W -\log w(i) + O(\log\log h)$; and
    \item for each distinct $i,j\in\{1,\ldots,h\}$, 
    \[   A(\alpha(i),\alpha(j)) 
    = \begin{cases}
       1 & \text{if $j=i+1$} \\
       -1 & \text{if $j=i-1$} \\
       \perp & \text{otherwise}
      \end{cases}
      \]
    \end{compactenum}
\end{lem}


\begin{proof}
  Define $w':\{1,\ldots,h\}\to \R^+$ as $w'(i)=w(i)+W/h$ and let $W':=\sum_{i=1}^h w'(i)=2W$.
  Using \lemref{biased-bst}, construct a biased binary search tree $T$ on $\{1,\ldots,h\}$ using $w'$ so that 
  \[   
    d_T(i)\le\log (2W)-\log(w(i)+W/h) \le \log W-\log w(i)+1 \enspace .
  \]
  Also
  \[
  d_T(i)\le\log (2W)-\log(w(i)+W/h) \le \log W-\log (W/h)+1 \le \log h + 1\enspace .
  \]
  for each $i\in\{1,\ldots,h\}$.  The code $\alpha(i$) for $i$ consists of three parts.  The first part, $\gamma(|\sigma_T(i)|)$, encodes the length of the path from the root to $i$ in $T$. The second part $\sigma_T(i)$ encodes the left/right turns along this path.  
  
  The third part $\delta(i)$ of $\alpha(i)$ is the encoding implicit in \lemref{predecessor-encoding}.  That is $\delta(i)$ consists of
  a single bit indicating whether $i$ has a left-child in $T$ and, in case $i$ does have a left-child, an Elias encoding of the value $s=d_T(i-i)-d_T(i)-1$.  More precisely, $\delta(i)=0$ or $\delta(i)=1,\gamma(s)$.  The length of $\delta(i)$ is at most $1+O(\log(s+1))=O(\log\log h)$.

  The function $A$ is now given by a simple algorithm: Given $\alpha(i)$ and $\alpha(j)$ we extract and lexicographically compare $\sigma_T(i)$ and $\sigma_T(j)$.  Assume, for now that $\sigma_T(i)$ is lexicographically less than $\sigma_T(j)$ so that, by \obsref{lexicographic}, $i < j$.  Now using $\sigma_T(j)$ and $\delta(j)$, compute $\sigma_T(j-1)$.  If $\sigma_T(j-1)=\sigma_T(i)$ then output $1$, otherwise output $\perp$.
  In the case where $\sigma_T(i)$ is lexicographically greater than $\sigma_T(j)$ we proceed in the same manner, but reversing the roles of $i$ and $j$ and outputting $-1$ in the case where $\sigma_T(i-1)=\sigma_T(j)$.
\end{proof}

\section{Subgraphs of $P\boxtimes P$}
\seclabel{pxp}

We begin with the special case in which $G$ is an $n$-vertex subgraph of $P_1\boxtimes P_2$ where $P_1=1,\ldots,m$ and $P_2=1,\ldots,h$ are paths. Each vertex of $G$ is a point $(x,y)\in\{1,\ldots,m\}\times \{1,\ldots,h\}$ in the $m\times h$ grid-with diagonals.  

% \pnote{TODO: Annoyingly, I now have to reverse the roles of $x$ and $y$ so that, in $H\boxtimes P$, $x\in V(H)$ and $y\in V(P)$. }

\subsection{The Labelling Scheme}

For each $y\in\{1,\ldots,h\}$, we let $L_y=\{x\in\{1,\ldots,m\}:(x,y)\in V(G)\}$.  The idea behind our approach is to have a sequence $T_1,\ldots,T_h$ of balanced binary search trees where, for each $y\in\{1,\ldots,h\}$,
\begin{enumerate}[(PR1)]
  \item $T_y$ contains (a superset of) $\bigcup_{b\in\{0,1\}} (L_{y+b}\cup \{x-1:x\in L_{y+b}\})$;
  
  \item $h(T_y)\le\log |T_y| + o(\log n)$;
  
  \item $W:=\sum_{y=1}^h |T_y| = O(n)$;
  
  \item there is a function $B:(\{0,1\}^*)^2\to\{0,1\}^*$ such that, for each $y\in\{x\in L_y\cap L_{y+1}$ there exists a binary string $\nu_y(x)$ of length $o(\log n)$ such that $B(\sigma_{T_{y}}(x), \nu_y(x))=\sigma_{T_{y+1}}(x)$.
\end{enumerate}

Property~(PR4) says that, for $x\in L_y\cap L_{y+1}$, $\sigma_{T_{y+1}}(x)$ can be obtained from $\sigma_{T_{y}}(x)$ with an additional $o(\log n)$ bits that we denote by $\nu_x(y)$.  Before describing the sequence of trees satisfying (PR1)--(PR4), we show how these can be used in a labelling scheme for $G$. 


\subsubsection{The Labels}

Given $G$ and $T_{1},\ldots,T_h$ we use a \lemref{row-code} with the weight function $w(y):=|T_y|$ to construct a code $\alpha:\{1,\ldots,h\}\to\{0,1\}^*$ where
\[  
  |\alpha(y)| = \log W-\log|T_y| + O(\log\log h) = \log n - \log|T_y| + O(\log\log n)
\]
for each $y\in\{1,\ldots,h\}$.  

Each vertex $z=(x,y)\in V(G)$ receives a label consisting of the following:  
\begin{enumerate}[(GC1)]
  \item $\alpha(y)$;
  \item $\gamma(|\sigma_{T_y}(x)|)$ and $\sigma_{T_y}(x)$;    
  \item $\delta_{T_y}(x)$;
  \item $\nu_y(x)$; and
  \item an array $a(v)$ of $8$ bits indicating whether each of the edges between $(x,y)$ and $(x\pm 1,y\pm 1)$ are present in $G$.  (Note that some of these 8 vertices may not even be present in $G$ in which case the resulting bit is set to 0 since the edge is not present in $G$.)
\end{enumerate}
The two major components of this label are $\alpha(y)$ (GC1) and $\sigma_{T_y}(x)$ (GC2) which, together have length $\log n + O(\log\log n)$.  The remaining components have size $O(|\nu_y(x)|+\log\log n)$.  

A large part of the remaining work involves describing $\nu_y(x)$ and, ultimately, showing that $|\nu_y(x)|\in O(\sqrt{\log n\log\log n})$. First, though, we show how (GC1)--(GC5) can be used for adjacency testing.

\subsubsection{Adjacency Testing}

Given the labels of $z_1=(x_1,y_1)$ and $z_2=(x_2,y_2)$ we can test if they are adjacent as follows: Using \lemref{row-code} with $\alpha(y_1)$ and $\alpha(y_2)$, determine which of the following applies:
\begin{enumerate}
  \item $|y_1-y_2|\ge 2$: In this case we immediately conclude that $z_1$ and $z_2$ are not adjacent since $y_1y_2\not\in E(P_1)$.  
  
  \item $y_1=y_2$: In this case, let $y:=y_1=y_2$, let $T:=T_y$ and lexicographically compare $\sigma_T(x_1)$ and $\sigma_T(x_2)$ to determine (without loss of generality) that $x_1<x_2$.  Using $\sigma_{T}(x_2)$ and $\delta_{T}(x_2)$, compute $\sigma_T(x_2-1)$.  If $\sigma_T(x_2-1)\neq \sigma_T(x_1)$ then immediately conclude that $z_1$ and $z_2$ are not adjacent, since $x_1x_2\not\in E(P_2)$.  Otherwise, we know that $x_1=x_2-1$ and $y_1=y_2$ so use the relevant bit of $a(z_1)$ (or $a(z_2)$) to determine if $z_1$ and $z_2$ are adjacent in $G$.
  
  \item $y_1=y_2-1$: In this case, use $\sigma_{T_{y_1}}(x_1)$ and $\nu_{y_1}(x_1)$ to compute $\sigma_{T_{y_2}}(x_1)$.  Now let $y:=y_2$, let $T:=T_{y}$, and proceed as in the previous case (but consulting a different bit of $a(z_1)$ in the last step.)
  
  \item $y_2=y_1-1$: In this case, use $\sigma_{T_{y_2}}(x_2)$ and $\nu_{y_2}(x_2)$ to compute $\sigma_{T_{y_1}}(x_2)$.  Now let $y:=y_1$, $T:=T_{y}$, and proceed as in the previous case (but consulting a different bit of $a(z_1)$ in the last step.)
\end{enumerate}


\subsection{The Tree Sequence $T_1,\ldots,T_h$}

All that remains is to find a sequence of binary search trees $T_1,\ldots,T_h$ with properties (PR1)--(PR4). This is a data structuring problem and we use a combination of existing and new techniques from data structures to solve it.

\subsubsection{Fractional Cascading}

Ultimately, we want to reach a point where, for each $y\in\{1,\ldots,h-1\}$ and each $x\in V(T_y)\cap V(T_{y+1})$, the difference between $\sigma_{T_y}(x)$ and $\sigma_{T_{y+1}}(x)$ has a description $\nu_y(x)$ of length $o(\log n)$.  In order for this to be possible, we require that $V(T_{y+1})$ not be wildly different from $V(T_{y})$.  In the world of data structures, this is achieved by the technique of \emph{fractional cascading} \cite{chazelle.guibas:fractional1, chazelle.guibas:fractional2, vaishnavi.wood:rectilinear}.

For non-empty sets $X,Y\subset \R$ and an integer $a$, we say that $X$ \emph{$a$-chunks} $Y$ if, for any $a+1$-element subset $S\subseteq Y$, there exists $x\in X$, such that $\min(S)\le x\le \max(S)$. Observe that, if $X$ $a$-chunks $Y$, then $|Y|\le a(|X|+1)\le 2a|X|$.

For each $y\in\{1,\ldots,h\}$, let $L^-_y:=L_y\cup \{x-1:x\in L_y\}$ and observe that $|L^-_y|\le 2|L_y|$, so $\sum_{x=1}^h|L^-_y| \le 2n$.  The following lemma is a form of fractional cascading.  A proof of (a much more general version of) this lemma is implicit in the iterated search structure of Chazelle and Guibas \cite{chazelle.guibas:fractional1}.   \snote{Put a self-contained proof in the appendix?}

\begin{lem}\lemlabel{fractional}
  There exists universal constants $a,b\ge 1$ and sets $V_1,\ldots,V_m$ such that
  \begin{compactenum}
    \item for each $y\in\{1,\ldots,m\}$, $V_y\supseteq L^-_y$;
    \item for each $y\in\{1,\ldots,m-1\}$, $V_y$ $a$-chunks $V_{y+1}$ and $V_{y+1}$ $a$-chunks $V_y$; and
    \item $\sum_{y=1}^m |V_y|\le bn$.
  \end{compactenum}
\end{lem}

Condition~2 in \lemref{fractional} has the following interpretation:  If we sort $V_y\cup V_{y+1}$, then we never see more than $a$ consecutive elements that only belong only to $V_y$ nor do we ever see more than $a$ consecutive elements that belong only to $V_{y+1}$.  Note that, if $x\in L_y\subseteq V_y$, then $x\in L^-_{y+1}\subseteq V_{y+1}$ so any $x\in L_y$ will appear as a node in both $T_y$ and $T_{y+1}$.

We will use \lemref{fractional} to create the sequence of binary search trees $T_1,\ldots,T_h$ where $V(T_y):=V_y$ for each $y\in\{1,\ldots,m\}$.  Condition~2 in \lemref{fractional} is what will ultimately make it possible to have $T_y$ and $T_{y+1}$ sufficiently similar so that, for any $x\in L_y$, there will be an $o(\log n)$-bit description $\nu_y(x)$ of how to modify $\sigma_{T_y}(x)$ in order to obtain $\sigma_{T_{y+1}}(x)$.  To begin with, we describe the three operations, bulk insertions, bulk deletions, and rebalancing that transform $T_y$ into $T_{y+1}$.  

\subsubsection{Bulk Insertions}

\begin{lem}\lemlabel{chunked-addition}
  For any binary search tree $T$ and any set $I\subset\R\setminus V(T)$ such that $V(T)$ $a$-chunks $I$, there exists a binary search tree $T'$ with $V(T')=V(T)\cup I$ that is a supergraph of $T$ and such that $h(T')\le h(T)+1+\log a$.
\end{lem}

\begin{proof}
  Let $z_0,\ldots,z_{|T||}$ denote the external nodes of $T$.  For each $i\in\{0,\ldots,|T|\}$, let $I_i:=\{x\in I: P_T(x)=z_i\}$.
  Since $V(T)$ $a$-chunks $I$, $|I_i|\le a$ for each $i\in\{0,\ldots,|T|\}$. For each $i\in\{0,\ldots,|T|\}$, construct a perfectly balanced binary search tree $T_i$ with $V(T_i):=I_i$ (so $h(T_i)\le\log|I_i|\le\log a)$). For each $i\in\{1,\ldots,|T|\}$, replace $z_i$ with $T_i$ in $T^+$ and call the resulting binary search tree $T'$.  Clearly $T'$ is a supergraph of $T$ and $h(T')\le h(T^+)+\log a\le h(T)+1+\log a$.
\end{proof}

\lemref{chunked-addition} is useful because the set of values $I_y:=V_{y+1}\setminus V_y$ that are present in $T_{y+1}$ but not in $T_y$ are $a$-chunked by $V_y$. Therefore, we can perform all the insertions required to move from $T_y$ to $T_{y+1}$ in such a way that $\sigma_{T_y}(x)=\sigma_{T_{y+1}}(x)$ for every $x\in V_y\cap V_{y+1}$ and so that $h(T_{y+1})\le h(T_y)+1+\log a$.  In short, this operation preserves existing labels without increasing the height by more than a constant.

\subsubsection{Bulk Deletions}

\begin{lem}\lemlabel{deletion-prefix}
  For any binary search tree $T$ and any $D\subseteq V(T)$, there exists a binary search tree $T'$ with $V(T')=V(T)\setminus D$ such that, for every $x\in V(T')$, $\sigma_{T'}(x)$ is a prefix of $\sigma_{T}(x)$.
\end{lem}

\begin{proof}
  It suffices to prove this for a singleton set $D$ containing one element $z$ since, for larger $D$ we can remove the values from $T$ one at a time.  The correctness of this follows from the transitivity of the ``is a prefix of'' relation on strings.
  
  The proof is by induction on $|T|$. If $|T|=1$, then the claim is vacuous since $V(T')=\emptyset$.  Therefore assume $|T|\ge 2$.  If $z$ is leaf in $T$, then use $T'=T-\{z\}$ and the result is trivial: $\sigma_{T'}(x)=\sigma_T(x)$ for every $x\in V(T')$.  
  
  Otherwise, $z$ has at least one child.  Without loss of generality, assume $z$ has a non-empty left subtree $T_l$.  Then the largest node $z'$ in $V(T_l)$ is the largest value in $V(T)$ that is less than $z$. Inductively remove $z'$ from $T_l$ to obtain $T_l'$ and make this the left child of $z$.  Now replace $z$ with $z'$ to obtain $T'$.  
  
  That $T'$ is indeed a binary search tree is straightforward to verify.  All that remains is to prove that $\sigma_{T'}(x)$ is a prefix of $\sigma_T(x)$ for every $x\in V(T')$. There are three cases:
  \begin{enumerate}
    \item $x=z'$: Since $z$ is an ancestor of $z'$ in $T$, $\sigma_{T'}(x)=\sigma_T(z)$ is a prefix of $\sigma_T(x)=\sigma_T(z')$.  
    
    \item $x\in V(T)\setminus\{z\}\setminus V(T_l)$: In this case $\sigma_T(x)=\sigma_{T'}(x)$.  
    
    \item $x\in V(T_l')$: In this case, $\sigma_{T'}(x)=\sigma_{T}(z),0,\sigma_{T_l'}(x)$ and $\sigma_T(x)=\sigma_{T}(z),0,\sigma_{T_l}(x)$.  By induction, $\sigma_{T_l'}(x)$ is a prefix of $\sigma_{T_l}(x)$, so $\sigma_{T'}(x)$ is a prefix of $\sigma_T(x)$. \qedhere
  \end{enumerate}
\end{proof}

\lemref{deletion-prefix} is useful because it means that, for any particular node of $x$ of $T_y$, the net effect of deleting all the elements of $D_y:=V_{y}\setminus V_{y+1}$ on $\sigma_{T_y}(x)$ can be encoded by a single integer $h_{T_y}(x)-h_{T_{y+1}}(x)\in\{0,\ldots,h(T_y)\}$.  Using \lemref{elias}, this requires only $O(\log h(T_y))$ bits.  If $T$ is at-all balanced, this is only $O(\log\log n)$ bits.

\subsubsection{Rebalancing}

\lemref{deletion-prefix} and \lemref{chunked-addition} help control the structure and height of the trees $T_1,\ldots,T_h$.  However, by themselves, they are not enough to ensure that $h(T_y)=\log|T_y|+o(\log n)$, even if $T_1$ is perfectly balanced.  Indeed, \lemref{deletion-prefix} does not ensure that the height of $T$ decreases even after many deletions.  Similarly, \lemref{chunked-addition} allows the height of $T$ to increase by $1+\log a$ even in the case where only $a$ additional elements are inserted.  

In this section, we introduce what (to our knowledge) is a novel form of rebalancing that ensures that $h(T_y)=\log|T_y|+o(\log n)$.  Standard forms of rebalancing, including height-balancing, weight-balancing, and partial rebuilding seem not to be applicable in our setting because we have unusual requirements.  In particular, the requirement $h(T_y)=\log|T_y|+o(\log n)$ is quite stringent. Most standard binary search trees guarantee only $h(T)\le c\log |T|$ for some some constant $c>1$.  Even more exotic is that the updates come in linear-sized batches, where batch $y$ consists of the insertions $I_y:=V_{y+1}\setminus V_y$, and the deletions $D_y:=V_y\setminus V_{y+1}$ and we require that there exists a simple relationship between $\sigma_{T_y}(x)$ and $\sigma_{T_{y+1}}(x)$ for each $x\in V_y\cap V_{y+1}$.     

We begin with the following simple lemma:

\begin{lem}\lemlabel{split}
  Let $T$ be a binary search tree and let $x$ be any node of $T$.  Then there exists a binary search tree $T'$ with $V(T'):=V(T)$,  $h(T')\le h(T)+1$,  whose root is $x$ and such that, for each node $z\in V(T')$, $\sigma_{T'}(z)$ is obtained by replacing a prefix of $\sigma_{T}(z)$ with a binary string from the set $\Pi:=\{\varepsilon\}\cup \bigcup_{i=0}^{h(T)}\{01^i, 10^i, 01^i0, 10^i1\}$.  
\end{lem}

\begin{proof}
Let $P_T(x)=x_0,\ldots,x_r$ be the root-to-$x$ path in $T$.  Partition $x_0,\ldots,x_{r-1}$ into two subsequence $a_1,\ldots,a_t$ and $b_1,\ldots,b_s$ consisting the values less than $x=x_r$ and greater than $x$, respectively.  

Refer to \figref{split}. Make a binary search tree $T_0$ that has $x$ as root, the path $a_0,\ldots,a_t$ as the left subtree of $x$ and the path $b_0,\ldots,b_s$ as the right subtree of $x$.  Note that $a_{i+1}$ is the right child of $a_i$ for each $i\in\{1,\ldots,t-1\}$ and $b_{i+1}$ is the left child of $b_i$ for each $i\in\{1,\ldots,s-1\}$. 

Next, consider the forest $F:=T-\{x_0,\ldots,x_r\}$. This forest consist of $r+2$ (possibly empty) trees $A_1,\ldots,A_{r-1},L,R$ where $L$ and $R$ are the subtrees of $T$ rooted at the left and right child of $x$ and, for each $i\in\{1,\ldots,r-1\}$, $A_i$ is the subtree of $T$ rooted at the child $c_i\neq x_{i+1}$ of $x_i$.  To obtain $T'$, replace each of the $r+2$ external nodes of $T_0^+$ with the appropriate tree in $F$.

\begin{figure}
  \begin{center}
    \includegraphics{figs/split-1} \\[1ex]
    $\Downarrow$ \\[1ex]
    \includegraphics{figs/split-2}
  \end{center}
  \caption{Moving the value $x$ to the root of a binary search tree $T$.}
  \figlabel{split}
\end{figure}

To see that $h(T')\le h(T)+1$ observe that this operation increases the depth of any node by at most one.  Indeed, for any node $z\in V(T')$, $V(P_{T'}(z))\subseteq V(P_T(z))\cup\{x\}$, so $|P_{T'}(z)|\le |P_T(z)|+1$ and therefore $d_{T'}(z) = |P_{T'}(z)|-1 \le |P_{T}(z)| = d_T(z)+1$. 

Next we argue that the stated relationship holds between $\sigma_T(x)$ and $\sigma_{T'}(x)$ for each $x\in V(T)$.  First observe that $\sigma_{T'}(x)=\varepsilon$, that $\sigma_{T'}(a_i)=01^i$, and that $\sigma_{T'}(b_i)=10^i$.  Therefore, for each $z\in\{x_0,\ldots,x_r\}$, $\sigma_{T'}(z)$ is obtained by replacing all of $\sigma_T(z)$ with a string in $\Pi$.

Any node $z\not\in\{x_0,\ldots,x_r\}$, belongs to $V(A)$ for some tree $A$ in the forest $F$.  Let $r_z$ denote the root of $A$.  The root-to-$z$ path in $T$ has the form $P_{T}(z)=x_0,\ldots,x_j,P_{A}(z)$ for some $j\in\{0,\ldots,r\}$.  The root-to-$z$ path in $T'$ has the form $P_{T'}(x_j),P_{A}(z)$.  Therefore $\sigma_{T'}(z)$ is obtained by removing the first $j+1$ bits of $\sigma_T(z)$ and replacing them with $\sigma_{T'}(x_j)$ followed by a 0 or a 1.  Since $x_j\in\{a_1,\ldots,a_t\}\cup\{b_1,\ldots,b_s\}$, $\sigma_{T'}(x_j)$ has the form $01^i$ or $10^i$ for some $i\in\{0,\ldots,h\}$.  Therefore, $\sigma_{T'}(z)$ is obtained by replacing a prefix of $\sigma_T(z)$ with an element of $\Pi$.
\end{proof}

\lemref{split} allows us to perform a perfect split at the root of $T$ and still maintain an $O(\log h(T))$-length description of the changes required to obtain $\sigma_{T'}(z)$ from $\sigma_T(z)$.  A useful fact about \lemref{split} is that the left and right subtree of $T'$ each have height at most $h(T)$.
The following lemma uses \lemref{split} recursively:

\begin{lem}\lemlabel{mini-split}
  Let $T$ be a binary search tree and let $x_1<\cdots<x_c$ be a set of nodes in $T$. Then there exists a binary search tree $T'$ with $V(T')=V(T)$, $h(T')\le h(T)+1+\log c$, and in which $T'[\{x_1,\ldots,x_c\}]$ is a tree of height at most $\log c$ that contains the root of $T'$ and such that, for each node $z\in V(T')$, there is a binary string of length $O((1+\log c)\log h(T))$ that describes how to convert $\sigma_T(z)$ into $\sigma_{T'}(z)$.  Furthermore, each tree in the forest $T'-\{x_1,\ldots,x_c\}$ is a tree of height at most $h(T)$.
\end{lem}

\begin{proof}
  The proof is by induction on $c$.  If $c=0$, the statement is trivially true with $T':=T$.  If $c\ge 1$,  apply \lemref{split} to $x_{\lceil c/2\rceil}$ to obtain a tree $T_\times$ rooted at at $x_{\lceil c/2\rceil}$.  Next apply induction on the left subtree of $T_\times$ with the set $x_1,\ldots,x_{\lceil c/2\rceil -1}$ and apply induction on the right subtree of $T_\times$ with the set $x_{\lceil c/2\rceil +1},\ldots,x_c$.  Let $T'$ be the tree whose root is $x_{\lceil c/2\rceil}$ and whose left and right subtrees are the result of these two applications of induction.
  
  It is straitforward to check that $h(T')\le h(T)+1+\log c$ and that $T[\{x_1,\ldots,x_c\}]$ is a tree of height at most $\log c$ that contains the root of $T'$.  The condition involving $\sigma_T(z)$ and $\sigma_{T'}(z)$ follows from the fact that any node $z\in V(T)$ is included in a subtree that is involved in at most $\log c$ inductive calls and each such call involves one application of \lemref{split}.  The effect of each application of \lemref{split} on $\sigma_T(z)$ can be recorded by $O(\log h(T))$ bits for a total of $O((1+\log c)\log h(T))$ bits.  Finally, the condition involving the heights of trees in $T'-\{x_1,\ldots,x_c\}$ follows from the fact that the depth of any node $z$ only increases (by 1) for each element of $x_1,\ldots,x_c$ on the path from the root of $T'$ to $y$.
\end{proof}

\begin{lem}\lemlabel{multi-split}
  Let $k,c\ge 1$ be integers, and let $T$ be a binary search tree.  Then there is a binary search tree $T'$ with $V(T')=V(T)$ in which
  \begin{compactenum}
    \item $h(T')\le h(T)+1$; 

    \item  each of the subtrees $T_i'$, $i\in\{1,\ldots,m'\}$ rooted at a depth-$(k+1)$ node of $T'$ has size at most $|T|/2^k$; and
    
    \item for each $x\in V(T)$, there exists an $O(k\log h(T))$-bit description of the changes required to create $\sigma_{T'}(x)$ from $\sigma_T(x)$.\snote{Use $B:(\{0,1\}^*)^2\to\{0,1\}^*$ terminology?}
  \end{compactenum}
\end{lem}

\begin{proof}
  Let $Z\subset V(T)$ be the set of at most $2^k-1$ nodes of $T$ of depth less than $k$.  Then $T-Z$ is a forest consisting of $m\le 2^{k}$ trees $T_1,\ldots,T_h$.  Each such tree $T_i$ has height at most $h(T)-k$.
  
  Select the nodes $X:=\{x_1,\ldots,x_{2^k}-1\}$ of $T$ where each $x_j$ has rank $\lfloor j|T|/2^k\rfloor$ in the set $V(T)$.\footnote{For a finite $X\subset\R$, and $x\in\R$, the \emph{rank} of $x$ in $S$ is $|\{x'\in S: x'<x\}|$.}  For each $j\in\{1,\ldots,m\}$, apply \lemref{mini-split} to the subtree $T_j$ with the at most $2^k$ values $Y\cap V(T_j)$ to obtain a new subtree $T_{j}'$.  Let $T^@$ be the tree obtained by replacing each of $T_1,\ldots,T_m$ with $T_1',\ldots,T_m'$, respectively.
  
  Now, $Z\cup X$ has size at most $2(2^{k}-1)< 2^k-1$ and $T^@[Z\cup X]$ is a connected subtree that contains the root of $T^@$.  Construct the tree $T'$ by building a complete binary search tree $T_0$ of height at most $k$ with $V(T_0)=Z\cup X$ and then replacing each external node $z$ of $T_0^+$ with the appropriate tree $A$ from the forest $T^@-(Z\cup X)$.  Since $h(T_0)\le k$, $h(A)\le h(T)-k$, we have $h(T') \le h(T_0^+) + h(A) \le h(T_0)+1+ h(A) \le h(T)+1$.
  
  For the final condition, observe that, for each each node $x\in (Z\cup X)\cap V(T_i)$ we can obtain $\sigma_{T^@}(x)$ from an additional $O(k\log h(T))$ bits as described in \lemref{mini-split}. From $\sigma_{T^@}(x)$ we can obtain $\sigma_{T'}(x)$ by deleting a prefix whose length can be specified using $O(\log h(T))$ bits and replacing this with a sequence of at most $k+1$ bits describing a path from the root of $T^+_0$ to an external node.
\end{proof}

\subsubsection{The Dynamic Data Structure}

We can now describe the sequence of binary search trees $T_1,\ldots,T_h$ satisfying (PR1)--(PR4).  Recall that $V(T_y)=V_{y}$ where $V_1,\ldots,V_m$ are the sets described by \lemref{fractional}.  The easiest way to describe these trees is in terms of a binary search tree $T$ that supports two operations:
\begin{enumerate}
  \item \emph{bulk deletion}: in which a set $D:=V_{i}\setminus V_{i+1}$ of at most $|T|(1-2/a)$ values are removed from $T$. 
  % The set $V_{i+1}=V(T)\setminus D$ of remaining values $a$-chunks $D$.  
  % \pnote{NOTE: We don't use the $a$-chunking property, except for the lower-bound it gives on $|V_{i+1}|$.}
  \item \emph{bulk insertion}: in which a set $I:=V_{i+1}\setminus V_i$ of at most $2a|T|$ new values are added into $T$.  The set $I$ of newly added values are $a$-chunked by $V(T)$.
\end{enumerate}
First we consider the effect of a pair of bulk deletion/bulk insertion operations on the height of $T$.

By \lemref{deletion-prefix}, each bulk deletion does not increase $h(T)$.  Thus, if we start with a tree $T_1^\times$ and peform one bulk deletion and one bulk insertion then, 
by \lemref{chunked-addition}, the resulting tree $T_2$ has height
\[
   h(T_2) \le h(T_1^\times)+1+\log a \enspace .
\]
Thus, a single round of bulk insertion/bulk deletion does not increase the height of $T$ by more than $1+\log a$.  The trick is to devise a rebalancing scheme that only allows this to continue for $o(\log n)$ rounds before guaranteeing that the tree returns to a more balanced state.

We will use a rebalancing scheme that is parameterized by an integer parameter $k\in o(\log n)$ to be discussed shortly.  This scheme guarantees that, for each $y\in\{1,\ldots,\lceil\log|T_1|/(k-\log(2a))\rceil+1\}$, 
\begin{enumerate}[(B1)]
  \item $h(T_y)\le h(T_1) + (y-1)(2+\log a)$;
  \item each subtree of $T_y$ rooted at a node of depth $(y-1)(k+1)$ has size at most $|T_1|(2a/2^k)^{y-1}$.
\end{enumerate}

The invariants (B1) and (B2) actually provide two upper bounds on the height of $h(T_y)$.  Let
\[
   y^* := \left\lceil \frac{\log|T_1|}{k-\log(2a)}\right\rceil + 1
\]
and observe that
\[
   |T_{1}|\left(\frac{2a}{2^k}\right)^{y^*-1} \le 1 \enspace .
\]
So, by (B2) every subtree of $T_{y^*}$ of depth $(y^*-1)(k+1)$ has size at most 1.  A subtree of size 1 has height 0.  Therefore,
\begin{align*}
  h(T_{y^*}) & \le (y^*-1)(k+1) \\
  & = \frac{(k+1)\log|T_1|}{k-\log(2a)} + k+1 \\
  & \le \frac{(k+1)(\log |T_{y^*}| + (y^*-1)\log(2a))}{k-\log(2a)} + k+1 \\
  & \le \log|T_{y^*}| + O(k + k^{-1}\log |T_y|) \\
  & \le \log |T_{y^*}| + O(k + k^{-1}\log n)
\end{align*}

Let $r_1=h(T_1)-\log |T_1|$ and note that $|T_y|\ge |T_1|/(2a)^{i-1}$ so, for $i\in\{1,\ldots,i^*\}$, (B1) gives the upper bound
\begin{align}
     h(T_y) & \le h(T_1) + (y-1)(2+\log a) \nonumber \\
            &= \log|T_1| + (y-1)(2+\log a) + r_1 \nonumber \\
            &\le \log |T_y| + (y-1)(2+2\log a) + r_1 \nonumber \\
            &\le \log |T_y| + (y^*-1)(2+2\log a) + r_1 \nonumber \\
            &\le \log |T_y| + O(k^{-1}\log|T_y|) + r_1 \nonumber \\
            &\le \log |T_y| + O(k^{-1}\log n) + r_1 \enspace .
\end{align}

Therefore, if $r_1\in o(\log n)$ and $k\in\omega_{n}(1)$, then
\[
  h(T_y) \le \log |T_y| + O(k^{-1}\log n) 
\]
for all $y\in\{1,\ldots,y^*\}$. At the end of this process, the tree $T_{y^*}$ has height
\[
    h(T_{y^*})= \log |T_{y^*}| + O(k + k^{-1}\log n) := \log |T_{y^*}| + r_{y^*}
\]
where $r_{y^*}:=h(T_{y^*})-\log|T_{y^*}|\in O(k + k^{-1}\log n)$.  At this point we continue as if $T_{y^*}$ were the first tree in the sequence.  This is is valid since $r_{y^*}\in o(\log n)$.  This establishes Property~(PR2) for the sequence of trees $T_1,\ldots,T_h$.  

It remains is to describe the rebalancing scheme that guarantees (B1) and (B2).  The scheme is simple: For each $y\in\{1,\ldots,y^*-1\}$,
\begin{enumerate}[(S1)]
  \item Apply \lemref{multi-split} to each of the at most $2^{(y-2)(k+1)}$ subtrees of $T_y$ rooted at nodes of depth $(y-2)(k+1)$ to obtain a new tree $T_y'$.
  \item Perform the bulk insert operations in $I_y:=V_{y+1}\setminus V_y$ on $T_y'$ to obtain the new tree $T_y''$.
  \item Perform the bulk delete operations in $D_y:=V_{y}\setminus V_{y+1}$ on $T_y''$ to obtain the new tree $T_{y+1}$.
\end{enumerate}

The proof that these operations preserve invariants (B1) and (B2) is by induction on $y$.  For the base case $y=1$, both properties are trivial: (B1) asserts that $h(T_1)\le h(T_1)$ and (B2) asserts that the subtree of $T_1$ rooted at the root of $T_1$ has size at most $|T_1|$.

For the inductive step, we assume invariants (B1) and (B2) hold for $T_{y-1}$ and prove that they hold for $T_{y}$, $y\ge 2$.  First we establish invariant (B1) as follows:
\begin{align*}  
  h(T_y) & \le h(T_{y-1}'') & \text{(by \lemref{deletion-prefix})} \\
          & \le h(T_{y-1}') + 1 + \log a & \text{(by \lemref{chunked-addition})} \\
          & \le h(T_{y-1}) + 2 + \log a & \text{(by \lemref{multi-split})} \\
          & \le h(T_1) + (y-2)(2+\log a) + 2 + \log a & \text{(by (B1) for $T_{y-1}$)}\\
          & = h(T_1) + (y-1)(2+\log a) \enspace ,
\end{align*}
as required.

Next we establish (B2).  By (B2) applied to $T_{y-1}$, every subtree of $T_{y-1}$ rooted at a node of depth $(y-2)(k+1)$ has size at most $|T_{1}|(2a/2^k)^{y-2}$.  Step (S1) then ensures that every subtree of $T_{y-1}'$ rooted at a node of depth $(y-1)(k+1)$ has size at most $|T_{1}|(2a/2^k)^{y-2}/2^k$.  The bulk addition in (S2) increases the size of each subtree by a factor of at most $2a$, so every subtree of $T_{y-1}''$ rooted at a node of depth $(y-1)(k+1)$ has size at most $|T_{1}|(2a/2^k)^{y-1} $.  Finally, the bulk deletion in (S3) does not increase the size of any subtree, so every subtree of $T_{y}$ rooted at a node of depth $(y-1)(k+1)$ has size at most $|T_{1}|(2a/2^k)^{y-1}$, as required.


\subsubsection{The Transition Labels}

In the previous section we established that the sequence $T_1,\ldots,T_m$ of trees satisfies Property~(PR2) since $|\sigma_{T_y}(x)|\le h(T_y)\le \log |T_y|+O(k+k^{-1}\log n)$.  These trees also satisfy Properties~(PR1) and (PR3) since, for each $y\in\{1,\ldots,h\}$, $V(T_y)=V_y$ is the set described in \lemref{fractional}.  Thus, all that remains is to describe the codes $\nu_y(x)$ for each $y\in\{1,\ldots,h-1\}$ and $x\in V_y\cap V_{y+1}$ that satisfy Property~(PR4).

The purpose of $\nu_y(x)$ is to provide instructions on how to modify $\sigma_{T_y}(x)$ to obtain $\sigma_{T_{y+1}}(x)$.  Therefore we consider the three steps (S1)--(S3) that convert $T_y$ into $T_{y+1}$.  
\begin{enumerate}[(S1)]
  \item applies \lemref{multi-split} to each subtree of $T_y$ at depth $(y-1)(k+1)$ to obtain a new $T_y'$.  Every node $x\in V(T_y)$ is contained in at most one such subtree $T_{y,x}$ that becomes a subtree $T_{y,x}'$ of $T_y'$.  For each such node $x$, we include the integer $(y-1)(k+1)$ in $\nu_y(x)$. \lemref{multi-split} describes $O(k\log h(T))=O(k\log\log n)$ bits that make it possible to recover $\sigma_{T_{y,x}'}(x)$ from $\sigma_{T_{y,x}}(x)$.  To get $\sigma_{T_y'}(x)$ we keep the first $(y-1)(k+1)$ bits of $\sigma_{T_y}(x)$ and append $\sigma_{T_{y,x}'}(x)$.
  
  \item applies \lemref{chunked-addition} (bulk insertion) to $T_{y}'$ to obtain $T_{y}''$.  This has no effect on any existing signature.  That is, $\sigma_{T_y'}(x)=\sigma_{T_{y}''}(x)$ for every $x\in V(T_i)$.
  
  \item applies \lemref{deletion-prefix} (bulk deletion) to $T_{y}''$ to obtain $T_{y+1}$. \lemref{deletion-prefix} shows that, for each $x\in V_{y+1}$, $\sigma_{T_{y+1}}(x)$ can be obtained by removing a suffix from $\sigma_{T_y''}(x)$.  For each $x\in V_{y+1}$, the length of this suffix, encoded using $O(\log\log n)$ bits is included in $\nu_{y}(x)$.
\end{enumerate}

This establishes, Property~(PR4) of $T_1,\ldots,T_h$ where each code $\nu_y(x)$ has length at most $O(k\log\log n)$. The optimal choice of $k=\left\lceil\sqrt{\log n/\log\log n}\right\rceil$, which results each vertex of the $n$-vertex subgraph $G$ of $P_1\boxtimes P_2$ being assigned a label of length $\log n + O(\sqrt{\log n\log\log n})$.

\begin{thm}
  There exists an adjacency-labelling scheme for subgraphs of $P_1\boxtimes P_2$ in which the vertices of any $n$-vertex subgraph $G$ of $P\boxtimes P$ are assigned labels of length at most $\log n + O(\sqrt{\log n\log\log n})$.
\end{thm}

\section{Subgraphs of $H\boxtimes P$}
\seclabel{hxp}

In this section we build on the result of \secref{pxp} to find labelling schemes for graphs $G$ that are subgraphs of $H\boxtimes P$ where $H$ is a $t$-tree and $P=1,2,\ldots,h$ is a path.  Our strategy is similar the approach taken in the previous section.
For each $y\in\{1,\ldots,h\}$ we define $L_y=\{x: (x,y)\in V(G)\}$ and we will build a labelling scheme for the induced graph $H[V_y]$ where $V_y\supseteq L_y$ that is based on a binary tree $T_y$ and such that the labels in this scheme have length $\log|T_y|+o(\log n)$.  In addition to this, we will use \lemref{row-code} to give each vertex $(x,y)\in V(G)$ a \emph{row label} of length $\log n - \log|T_y|+o(\log n)$.

\subsection{A Labelling Scheme for $t$-Trees}

We begin by describing a labelling scheme for $t$-Trees that, like the labelling scheme for paths, is based on a binary search tree.

\subsubsection{$t$-Trees}

We begin by describing a labelling schemes for $t$-trees. The ideas behind this scheme are not new; this is essentially the labelling scheme for $t$-trees described by Gavoille and Labourel \cite{gavoille.labourel:shorter}.  However, we present these ideas in a manner that makes it natural to generalize the results of \secref{pxp}.

A graph $H$ is a \emph{$t$-tree} if $H$ is a clique on $t$ vertices or if $H$ contains a vertex $v$ of degree $t$ whose neighbours form a clique and $H-v$ is a $t$-tree.  

Note that the recursive definition of $t$-trees implies that there is a vertex ordering $v_1,\ldots,v_{m}$ of $V(H)$ such that $v_1,\ldots,v_t$ form a clique and, for each $i\in\{t+1,\ldots,t_m\}$, $C_H(v_i):=B_H(v_i)\cap \{v_1,\ldots,v_{i}\}$ is a clique of size $t+1$.  We call $C_H(v_i)$ the \emph{family clique} of $v_i$ and $C_H(v_i)\setminus\{v_i\}$ the \emph{parent clique} of $v_i$.  The order $v_1,\ldots,v_m$ is called a \emph{construction order} for $H$.

The order $v_1,\ldots,v_m$ (in particular, the fact that each $v_i$ has at most $t$ neighbours among $v_1,\ldots,v_{i-1}$) implies that $V(H)$ has a proper $(t+1)$-colouring $\varphi:V(H)\to\{1,\ldots,t+1\}$.  For each $i\in\{t+1,\ldots,m\}$ and each $j\in\{1,\ldots,t+1\}$ the \emph{$j$-parent} $p_j(v_i)$ of $v_i$ is the unique element $p\in C_H(v_i)$ with $\varphi(p)=j$.  Note that $v_i$ is the $j$-parent of itself for exactly one $j\in\{1,\ldots,t+1\}$.

\subsubsection{Interval Graphs}

For real numbers $a\le b$, let $[a,b]:=\{ x\in\R: a\le x\le b\}$, and let
$\mathbb{I}:=\{[a,b]: a,b\in\R,\, a\le b\}$ denote the set of closed real intervals.  For a finite set $S\subset\mathbb{I}$ of intervals, the \emph{interval intersection graph} $G_S$ is the graph with vertex set $V(G_S):=S$ and in which the edge $vw\in E(I)$ if and only if $v\cap w\neq \emptyset$.  

% The \emph{thickness} $\omega(G_S)$ of $S$ is the size of its largest clique, which (by Helly's Theorem) is equal to $\max_{x\in\R}|\{v\in V(H):x\in v\}|$.  By Dilworth's Theorem (applied to the poset $(V(I),\prec)$ where $[a,b]\prec [c,d]$ iff $b<c$), the chromatic number $\chi(I)$ of $I$ is equal to its thickness, i.e., $\chi(I)=\omega(I)$.  

The following well-known result states that every $n$-vertex $t$-tree is the subgraph of an interval graph with thickness $O(t\log n)$ \snote{Help: reference?}:

\begin{lem}\lemlabel{interval-representation}
  For every $n$-vertex $t$-tree $H$, there exists a mapping $f:V(H)\to\mathbb{I}$, such that the interval intersection graph $G_S$ with $S:=\{f(v):v\in V(H)\}$ has thickness at most $\log_{3/2} t\log n$ and, for every $vw\in E(H)$, $f(v)\cap f(w)\neq\emptyset$.  
  
  Furthermore, for every proper $(t+1)$-colouring $\varphi:V(H)\to\{1,\ldots,t+1\}$ of $H$, there exists a proper colouring $\varphi':V(G_S)\to\{1,\ldots,\lceil\log_{3/2} n\rceil\}\times\{1,\ldots,t+1\}$ where, for each $v\in V(H)$, $\varphi'(f(v))=(j,\varphi(v))$ for some $j\in\{1,\ldots,\lceil\log_{3/2} n\rceil\}$.
\end{lem}

In light of \lemref{interval-representation} we will not distinguish between a vertex $v\in V(H)$ and the interval $f(v)$.  That is, we will treat the nodes of every $t$-tree as intervals that satisfy the conditions of \lemref{interval-representation}.

A point $x\in\R$ \emph{stabs} an interval $[a,b]$ if $\{x\}\cap [a,b]=\{x\}$. A finite set $X\subset\R^2$ of points \emph{stabs} a set $S\subset\mathbb{I}$ of intervals if, for every $[a,b]\in S$, at least one point $x\in X$ stabs $[a,b]$, i.e., $X\cap [a,b]\neq\emptyset$.

\begin{lem}\lemlabel{common-ancestor}
  Let $S\subset\mathbb{I}$ be a set of intervals, let $X\subset\R$ be a set of points that stabs $S$, and let $T$ be a binary search tree with $V(T):=X$, let $[a,b]\in S$ and let $x$ be the lowest common ancestor, in $T$, of $X\cap [a,b]$.  Then $x\in [a,b]$.
\end{lem}

\begin{proof}
  Since $x$ is the least common ancestor of $X\cap[a,b]$, either $x\in X\cap[a,b]$, in which case there is nothing prove, or there is some pair $x_1,x_2\in X\cap[a,b]$ such that $x_1$ is in the subtree of $T$ rooted at the left child of $x$ and $x_2$ is in the subtree of $T$ rooted at the right child of $x$.  By the binary search tree property, $x_1<x<x_2$. But $x_1,x_2 \in [a,b]$, so $a\le x_1<x<x_2\le b$, so $x\in [a,b]$.  
\end{proof}

\subsubsection{The Labelling Scheme}
\seclabel{t-tree-labelling}

We can use \lemref{common-ancestor} to create a labelling scheme for $t$-trees based on any binary search tree containing stabbing set.  Let $H$ be a $t$-tree whose vertex set $V(H):=S$ consists of the intervals described by \lemref{interval-representation}, let $v_1,\ldots,v_m$ be a construction order for $H$, let $\varphi:V(H)\to\{1,\ldots,t+1\}$ be a proper colouring of $H$, and let $\varphi':V(H)\to\{1,\ldots,\lceil\log_{3/2} n\rceil\}\times\{1,\ldots,t+1\}$ be the extension of $\varphi$ to a proper colouring of $G_S$ described in \lemref{interval-representation}.

Let $X\subset\R$ be any set of points that stab $S$, let $T$ be a binary search tree with $V(T)=X$, and let $\varphi:V(H)\to\{1,2,\ldots,\lfloor t\log n\rfloor\}$ be a proper colouring of the interval graph with vertex set $S$.

For each vertex $v:=[a_v,b_v]\in V(H)$, let $x_T(v)$ denote the lowest-common-ancestor of $X\cap [a_v,b_v]$ in $T$ (see \figref{x}).  For any subset $C\subseteq V(H)$, let $x_T(C):=\{x_T(x):x\in C\}$.  

\begin{figure}
  \begin{center}
    \includegraphics{figs/x}
  \end{center}
  \caption{The definition of $x_T(v)$.}
  \figlabel{x}
\end{figure}

\begin{lem}\lemlabel{one-path}
  For any $v\in V(H)$ with family clique $C_H(v)$, the set of nodes $x_T(C_H(v))$ are all contained a single root-to-leaf path in $T$.
\end{lem}

\begin{proof}
  Suppose for the sake of contradiction that this is not true, so there are $x_1,x_2\in x_T(C_H(v))$ neither of which is an ancestor of the other.  Then consider the lowest common ancestor $x$ of $x_1$ and $x_2$ in $T$.  Assume without loss of generality that $x_1$ is in the subtree of $T$ rooted at $x$'s left child and $v_2$ is in the subtree of $T$ rooted at $x$'s right child, so $x_1<x<x_2$. The node $x_1:=x_T(v_1)$ for some $v_1:=[a_1,b_1]\in C_H(v)$ and $x_2=x_T(v_2)$ for some $v_2:=[a_2,b_2]\in C_H(v)$.  Since $v_1$ and $v_2$ are both in the family clique $C_H(v)$, the edge $v_1v_2\in E(H)$, and therefore $[a_1,b_1]\cap[a_2,b_2]\neq\emptyset$.  This implies that $x\in [a_i,b_i]$ for at least one $i\in\{1,2\}$.  But this is a contradiction since $x_T(v_i)$ is supposed to be the lowest common ancestor of $[a_i,b_i]\cap X$.
\end{proof}

The following observation shows that a vertex $v$ of $H$ is uniquely identified by $x_T(v)$ and $\varphi'(v)$.

\begin{obs}\obslabel{unique-id}
    For any two distinct vertices $v,w\in V(H)$, $x_T(v)\neq x_T(w)$ or $\varphi'(v)\neq\varphi'(w)$.  Consequently, $\sigma_T(x_T(v))\neq \sigma_T(x_T(w))$ or $\varphi'(v)\neq\varphi'(w)$. 
\end{obs}

\begin{proof}
  If $x_T(v)=x_T(w)=x$, the intervals $v=[a_v,b_v]$ and $w=[a_w,b_w]$ each contain $x$, so $vw\in E(G_S)$.  Therefore $\varphi'(v)\neq\varphi'(w)$ since $\varphi'$ is a proper colouring of $G_S$.  The second, equivalent, statement of the observation is immediate from the fact that the $\sigma_T: V(T)\to\{0,1\}^*$ is injective, so $\sigma_T(x)$ uniquely identifies $x$.
\end{proof}

For each node $v\in V(H)$, let $\sigma_T(v)$ denote the path in $T$ that begins at the root of $T$ and ends at the node in $x_T(C_H(v))$ of maximum depth.  By \lemref{one-path}, $\sigma_T(v)$ contains every node in $x_T(C_H(v))$.
The label for a vertex $v\in V(H)$ consists of the following (we ignore any integers, such as $t$, $|\sigma_T(v)|$, and $\lceil\log_{3/2} m\rceil$, that can be encoded using \lemref{elias}):

\begin{enumerate}[(TC1)]
  \item $\sigma_T(v)$;
  \item $d_T(x_T(p_j(v))$ for each $j\in\{1,\ldots,t+1\}$; 
  \item $\varphi'(p_j(v))$ for each $j\in\{1,\ldots,t+1\}$; and
  \item $\varphi'(v)$.
\end{enumerate}

Given the labels of two vertices $v,w\in V(H)$, we test if $v$ and $w$ are adjacent as follows:
\begin{enumerate}[({A}1)]
  \item If the labels of $v$ and $w$ are identical then $v=w$, so return false.
  
  \item (uniquely identify $v$) From (TC4) extract $j:=\varphi(v)$ (which is contained in $\varphi'(v)$).  From (TC2) extract $d:=d_T(x_T(p_j(v)))$. Take the length-$d$ prefix of $\sigma_T(v)$ to get $\sigma_T(x_T(v))$.

  \item (check if $v$ is the $j$-parent of $w$) Extract $d:=d_T(x_T(p_j(w)))$ and take the length-$d$ prefix of $\sigma_T(w)$ to get $\sigma_T(x_T(p_j(w)))$.  If $\sigma_T(x_T(v))=\sigma_T(x_T(p_j(w)))$ then, by \obsref{unique-id}, $v$ is the $j$-parent of $w$ so $vw\in E(H)$, so return true.
  
  \item[(A4,A5)] Repeat (A2) and (A3) with the roles of $v$ and $w$ reversed.
  
  \item[(A6)] Return false
\end{enumerate}

The correctness of this procedure can be seen as follows:
\begin{itemize}
  \item Given the labels of $v$ and $w$ we can recover $\sigma_T(x_T(v))$, $\varphi'(v)$, $\sigma_T(x_T(w))$, and $\varphi'(w)$ so, by \obsref{unique-id}, the label of $v$ and the label of $w$ are identical if and only if $v=w$, so the negative result in (A1) is never incorrect;
  \item from \obsref{unique-id}, positive results in (A3) and (A5) are never incorrect; and 
  \item for every $vw\in E(H)$, there exists a $j\in\{1,\ldots,t+1\}$ such that $v$ is the $j$-parent of $w$ or $w$ is the $j$-parent of $v$, so the negative result in (A6) is never incorrect.    
\end{itemize}
In fact, this labelling scheme proves the slightly stronger result:

\begin{lem}\lemlabel{t-tree-labelling}
  Let $H$, $S$, $X$, and $T$ be defined as above.  There exists a function $\mathds{T}:(\{0,1\}^*)^2\to \Z\cup\{\perp\}$ such that there is a prefix-free code $\tau_H:V(H)\to\{0,1\}^*$ where $|\tau_H(v)|=h(T) + O(t(\log t + \log h(T)))$ for every $v\in V(H)$ and, for every $v,w\in V(H)$, 
  \[
      \mathds{T}(\tau_H(v),\tau_H(w)) = \begin{cases}
      0 & \text{if $v=w$} \\
      -j & \text{if $v$ is the $j$-parent of $w$} \\
      j & \text{if $w$ is the $j$-parent of $v$} \\
      \perp & \text{otherwise.}
    \end{cases}
  \]
\end{lem}

\begin{proof}
  Most of the details of this labelling are described above, though we have thus far ignored the length of the labels, which we analyze now.
  
  Part (TC1) of each label has length $|\sigma_T(v)|\le h(T)$.  For each $x\in V(T)$, $d_T(x)\le h(T)$, so Part~(TC2) of each label requires $O(t\log h(T))$ bits.  Part~(TC3) of each label requires $O(t\log t + t\log h(T))$ bits.  Part~(TC4) of each label requires $O(\log t + \log h(T))$ bits.
\end{proof}

Taking $X$ to be a minimal set that stabs $S$ (so $|X|\le |S|=m$) and taking $T$ to be a perfectly balanced binary search tree (so $h(T)\le \log|X|\le\log m$) we obtain the following corollary.

\begin{cor}\corlabel{t-tree-labelling}
  There exists a function $\mathds{T}:(\{0,1\}^*)^2\to \Z\cup\{\perp\}$ such that, for any $m$-vertex $t$-tree $H$ there is a prefix-free code $\tau_H:V(H)\to\{0,1\}^*$ where $|\tau_H(v)|=\log m + O(t(\log t + \log\log m))$ for every $v\in V(H)$ and, for every $v,w\in V(H)$, 
  \[
      \mathds{T}(\tau_H(v),\tau_H(w)) = \begin{cases}
      0 & \text{if $v=w$} \\
      -j & \text{if $v$ is the $j$-parent of $w$} \\
      j & \text{if $w$ is the $j$-parent of $v$} \\
      \perp & \text{otherwise.}
    \end{cases}
  \]
\end{cor}


\subsection{A Dynamic Data Structure and Transition Labels}

We point out again that the result in \corref{t-tree-labelling} is not new and the labelling scheme is just a reformulation of the one given by Gavoille and Labourel \cite{gavoille.labourel:shorter} that happens to be convenient for what we are going to do next. Most important for us is that the largest part of the codes come from paths in the binary search tree $T$ that contains the stabbing set $X$.  

What we will do next is to show that the solution presented in \secref{pxp} generalizes to the current setting.  The only additional complication comes from the fact that each node $x$ in the binary search tree $T$ is equipped with a collection $B_x:=\{v\in V(H):x_T(v)=x\}$ of intervals.  Any structural changes that we make to $T$ may result in changes to $B_x$, which result in changes to the labels for the vertices of $H$ that enter or leave $B_x$.  We must show that these changes can be encoded using few bits.  We now proceed.

Let $G$ be an $n$-vertex subgraph of $H\boxtimes P$ where $H$ is a $t$-tree and $P=1,\ldots,h$ is a path. For each $y\in\{1,\ldots,h\}$, let $S_y=\{v\in V(H): (v,y)\in V(G)\}$ \snote{Consistency: $S_y$ was $L_y$ in \secref{pxp}} and let $S^-_y=\bigcup_{j=1}^{t+1}\{p_j(v):v\in S_y\}$.  Note that $|S^-_y|\le t|S_y|$. For each $y\in\{1,\ldots,t+1\}$, let $X_y\subset\R$ be the set of all endpoints of intervals in $S^-_y\cup S^-_{y-1}$.

Now apply the result of \secref{pxp} to get a sequence of trees $T_1,\ldots,T_h$ where, for each $y\in\{1,\ldots,m\}$, 
\begin{enumerate}[(PRX1)]
  \item $V(T_y)\supseteq X_y$;
  \item $h(T_y)\in \log |T_y| + o(\log n)$;
  \item $W:=\sum_{y=1}^h |T_y|\in O(n)$; and
  \item There is a function $B:(\{0,1\}^*)^2\to\{0,1\}^*$ such that, for every $x\in V(T_y)\cap V(T_{y+1})$, there is an $o(\log n)$-bit string $\nu_y(x)$ such that $B(\sigma_{T_y}(x),\nu_y(x))=\sigma_{T_{y+1}(x)}$.
\end{enumerate}

As before, we can use \lemref{row-code} to assign each vertex $v=(x,y)\in V(G)$ a label $\alpha(v)$ of length $\log n - \log|T_y| + o(\log n)$. For any two nodes $v_1=(x_1,y_1)\in V(G)$ and $v_2=(x_2,y_2)$, $\alpha(v_1)$ and $\alpha(v_2)$ are sufficient to determine if $|y_1-y_2|\le 1$ and, if so, the actual value of $y_1-y_2$.

Now, for each $y\in\{1,\ldots,h\}$, $V(T_y)$ stabs $S^-_y\cup S^-_{y-1}$, so we can use $T_y$ to create a labelling scheme $\tau_y:(S^-_y\cup S^-_{y-1})\to\{0,1\}^*$ for the induced graph $H[S^-_y\cup S^-_{y-1}]$ as described in \secref{t-tree-labelling} leading to \lemref{t-tree-labelling} where the labels have length $h(T_y) + o(\log n)=\log|T_y|+o(\log n)$.  

For each $v\in S^-_y$, the label $\tau_y(v)$ has four parts (TC1)--(TC4).  Parts~(TC3) and (TC4) are completely determined by $H$ and $v$ and are independent of $T_y$.  In particular parts (TC3) and (TC4) of $\tau_{y_1}(v)$ and $\tau_{y_2}(v)$ are the same for any $y_1$ and $y_2$ such that $(v,y_1),(v,y_2)\in V(G)$.  

Part~(TC2) of $\tau_y(v)$, however depends very much on the tree $T_y$.  Luckily, Part~(TC2) is small: It has size $o(\log n)$, so the label of $(v,y)$ can include Part~(TC2) for $\tau_{y}(v)$ and $\tau_{y+1}(v)$.

The difficulty comes from Part~(TC1) of $\tau_y(v)$.  This part, $\sigma_{T_y}(v)$, is determined by the path from the root of $T_y$ to the deepest node $z\in x_{T_y}(C_H(v))$.  When $T_y$ is restructured to become $T_{y+1}$, the path defining $\sigma_{T_y}(v)$ changes, the relative depths of nodes in $x_{T_y}(C_H(v))$ may change and even the value $x_{T_y}(w)$ may change for any $w\in V(H)$.  

We now examine the three operations that define how $T_y$ is transformed into $T_{y+1}$ to show how the effects of these operations on the labels of nodes in $S^-_y\cap S^-_{y+1}$ can be encoded succinctly.  For each $v\in S_y$, the changes to $\sigma_{T_y}(v)$ required to recover $\sigma_{T_{y+1}}(v)$ will be encoded in a string $\mu_y(v)$.

Recall that the three operations that transform $T_y$ into $T_{y+1}$ are (S1)~rebalancing, (S2)~bulk insertion and (S3)~bulk deletion. Rebalancing takes $T_y$ onto a tree $T_y'$. Bulk insertion takes $T_y'$ onto a tree $T_y''$.  Bulk deletion takes $T_y''$ onto $T_{y+1}$.


\subsubsection{(S1): Rebalancing}

The rebalancing operation involves applying \lemref{multi-split} to all the subtrees of $T$ rooted at nodes of depth $(i-1)(k+1)$.  This, in turn, involves applying \lemref{split} to $2^k-1$ nodes so that they become roots of specific subtrees.

For a binary tree $T$ and a node $z\in V(T)$, let $P_T(z)$ denote the path from the root of $T$ to $z$.  We begin with a simple observation about the restructing operation in \lemref{split}.

\begin{obs}\obslabel{x-switch}
  Let $T$ be a binary search tree, let $S$ be a set of intervals, and let $T'$ be the result of applying \lemref{split} to $T$ to make some node $x'\in V(T)$ become the root.  Then, for each $v\in S$, $x_{T'}(v)\in\{x_T(v), x'\}$.
\end{obs}

\begin{proof}
  Consider any node $z\in V(T)$ and without loss of generality, assume $z>x'$. 
  If $x'$ is not in the subtree of $T$ rooted at $z$, then $P_T(z)=P_{T'}(z)$.  If $x'$ is in the subtree of $T$ rooted at $z$, then $P_T(z)$ is obtained by deleting all values less than $x'$ from $P_T(z)$ and (possibly) inserting $x'$.
  
  Now, consider any $v=[a,b]\in S$ and let $x:=x_T(v)$. By definition, $x$ is is the only node in $P_T(x)$ contained in $[a,b]$. Therefore, the only nodes of $P_{T'}(x)$ contained in $[a,b]$ are $x$ and (possibly) $x'$.  Therefore $x_{T'}(v)\in \{x,x'\}$.
\end{proof}

For any $v\in S^-_{y}\cap S^-_{y+1}$, and any $y\in\{1,\ldots,h\}$, let $P_{T_y}(v)=x_0,\ldots,x_r$ be the path from the root of $T_y$ to the deepest node in $x_{T_y}(C_H(v))$.  In other words, $\sigma_{T_y}(v)=\sigma_{T_y}(x_r)$ and $x_r=x_{T_y}(w)$ for some $w\in C_H(v)$.

A rebalancing operation applies \lemref{multi-split} to each subtree of $T_y$ rooted at a node of depth $(i-1)(k+1)$.  Refer to \figref{rebalance-t-tree}. At most one of these subtrees contains nodes from $P_{T_y}(v)$. We first establish that none of the other application of \lemref{multi-split} has any effect on $\sigma_{T_y}(v)$.  Let $T_y^0$ be the result of applying \lemref{multi-split} to every subtree rooted at a depth $(i-1)(k+1)$ node that does not contain any nodes of $P_{T_y}(v)$.  By \obsref{x-switch}, $x_{T_y}(u)=x_{T_y^0}(u)$ for all $u\in C_H(v)$.  The path $P_{T_y}(v)=x_0,\ldots,x_r$ is also a path in $T_y^0$ and this is the path in $T_y^0$ from the root to the deepest node in $x_{T_y^0}(C_H(v))$. So, by definition, $P_{T_y}(v)=P_{T_y^0}(v)$.

\begin{figure}
  \begin{center}
    \includegraphics{figs/rebalance-t-tree}
  \end{center}
  \caption{Applying \lemref{multi-split} to each of the depth-$(i-1)(k+1)$ subtrees of $T_y$ except $T_*$ has no effect on $P_{T_y}(v)$.}
  \figlabel{rebalance-t-tree}
\end{figure}

Let $T_y'$ be the result of applying \lemref{multi-split} to the subtree $T_*$ of $T_y^0$ that contains some nodes of $P_{T_y}(v)=P_{T_y^0}(v)=x_0,\ldots,x_r$ and let $T_*'$ denote the result of applying \lemref{multi-split} to $T_*$.
The subtree $T_*$ contains some suffix $x_i,\ldots,x_r$ of $x_0,\ldots,x_r$.  By definition $x_r=x_{T_y}(w)$ for some $w=[a_w,b_w]\in C_H(v)$.  Since the only restructuring done by \lemref{multi-split} is done by applications of \lemref{split} to subtrees contained in $T_*$,  \obsref{x-switch} implies that $x_{T_y'}(w)\in V(T_*)$.  Therefore $P_{T_y'}=x_0,\ldots,x_{(i-1)(k-1)-1},x'_0,\ldots,x'_{s}$ where $x_0',\ldots,x'_s$ is a path in $T_*'$. By including the integer $(i-1)(k+1)$ in $\mu_y(v)$, we can therefore focus on encoding the differences between $\sigma_{T_*}(x_r)$ and $\sigma_{T_*'}(x'_s)$.

To simplify notation, we now assume that $T^*=T_y$, so that the one application of \lemref{multi-split} that affects $P_{T_y}(v)$ is at the root of $T_y$.
This application of \lemref{multi-split} involves several applications of \lemref{split}.  Consider the first application of \lemref{split}, that takes some node $x'$ in $T_y$ and moves it to the root, creating a new tree $T_y^1$.

By \obsref{x-switch} $x_{T_y^1}(w)\in\{x',x_{T_y^1}(w)\}$ for every $w\in C_H(v)$.  If $x_{T_y^1}(w)=x'$ for every $w\in C_H(v)$, then $P_{T_y^1}(v)=x'$ is a path of length 0.  This can be recorded by adding $O(1)$ bits to $\mu_y(v)$. Otherwise, there is a non-empty subset $W\subset C_H(v)$ such that $x_{T_y^1}(w)=x_{T_y}(w)$ for each $w\in W$.  Let $w^*\in W$ be an element that maximizes $d_{T_y}(w^*)$.  We claim that $w^*$ also maximizes $d_{T_y^1}(w^*)$.  Indeed, as noted already in the proof of \obsref{x-switch} $P_{T_y^1}(w^*)$ consists of $x'$ followed by a subsequence of $P_{T_y}(w^*)$ that includes every node in $W$.  Therefore, $P_{T_y^1}(w^*)$ contains every node of $W$ and ends at $w^*$.  Therefore $d_T(w^*)\ge d_T(w)$ for all $w\in W$.

Therefore $P_{T_y^1}(v)=P_{T_y^1}(w^*)$.  Now $w^*$ appear in $P_{T_y}(v)$, so $\sigma_{T_y}(w^*)$ can be recovered from $\sigma_{T_y}(v)$ and $d_{T_y}(w^*)$, so the $O(\log\log n)$ bits required to encode $d_{T_y}(w^*)$ are included in $\mu_y(v)$.  Furthermore, as shown in \lemref{split}, $\sigma_{T_y^1}(w^*)=\sigma_{T_y^1}(v)$ can be recovered from $\sigma_{T_y}(w^*)$ and an additional $O(\log\log n)$ bits that are included in $\mu_y(v)$.  In this way, we can recover $\sigma_{T_y^1}(v)=\sigma_{T_y^1}(w^*)$ with the addition of only $O(\log\log n)$ bits to $\mu_y(v)$.

Now \lemref{multi-split} involves $2^k-1$ applications of \lemref{split} resulting in intermediate trees $T_y^1,\ldots,T_y^{2^k-1}=T_y'$.  However, these applications are partitioned into $k$ sets $R_0,\ldots,R_{k-1}$, where the applications of \lemref{multi-split} in $R_i$ operate on $2^i$ disjoint subtrees rooted at nodes of depth $i$.  In particular, only one subtree in each set $R_i$ contains any nodes from $P_{T}(v)$.  Each of these applications adds $O(\log\log n)$ bits to $\mu_y(v)$ for a total of $O(k\log\log n)$ bits.

Therefore, $\mu_y(v)$ contains a sequence of $O(k\log\log n)$ bits, that makes it possible to recover $\sigma_{T_y'}(v)$ from $\sigma_{T_y}(v)$ for every $v\in S_y$.

\subsubsection{(S2): bulk Insertions}

During the bulk insertion phase, elements in $I_y:=X_{y+1}\setminus X_y$ are added to $T_y'$ to create the new tree $T_{y}''$.  The newly added elements are added to small complete binary search trees that are attached to existing nodes of $T_{y}'$.  This ensures that, in $T_y''$, no vertex of $T_y$ has any ancestor in $I_y$.  This in turn ensures that $x_{T_y''}(v)=x_{T_y'}(v)$ for every $v\in S_y$.  

Now, for every $v\in S_y$, $C_H(v)\subseteq S^-_y$, so $P_{T_y'}(v)$ is well-defined.  Since $T_y''$ is a supergraph of $T_y'$ and $x_{T_y''}(w)=x_{T_y'}(w)$ for each $w\in C_H(v)$, $P_{T_y''}(v)=P_{T_y'}(v)$ and therefore $\sigma_{T_y''}(v)=\sigma_{T_y'}(v)$ for every $v\in S_y$.

\subsubsection{(S3): bulk Deletions}

Each individual deletion either involves removing a leaf from $T_y''$ or it involves (possibly repeatedly) replacing some value $x$ with some value $x'$ that either the smallest value in $x$'s right subtree or the largest value in $x$'s left subtree.

Let $v=[a_v,b_v]\in S_y\cap S^-_y$.  Recall that, by definition, $a_v$ and $b_v$ are both contained in $X_y$ and in $X_{y+1}$.  Refer to interval $v_1$ in \figref{deletion-t-tree}.  If $x_{T_y''}(v)=x$, then $x\in[a_v,b_v]$ and, since $a_v,b_v\in X_{y}$, $x'\in[a_v,b_v]$.  Therefore, if $x_{T''}(v)=x$ then, after removing $x$ and replacing it with $x'$, $x_{T''}(v)=x'$.  

The only other possible change of this type occurs when this replacement operation causes $x'$ to become an ancestor of some node $x''$ that it wasn't before.  See vertex $v_2$ in \figref{deletion-t-tree}.  In this case, some $v\in S_y$ with $x_{T_y''}(v)=x''$ before the replacement will have $x_{T_y''}(v)=x'$ after the replacement.

\begin{figure}
  \begin{center}
    \includegraphics{figs/deletion-t-tree-1} \\[1ex]
    $\Downarrow$ \\[1ex]
    \includegraphics{figs/deletion-t-tree-2}
  \end{center}
  \caption{A deletion in $T''_y$ can only remove a suffix from $\sigma_{T''_y}(x_{T''_y}(v))$.}
  \figlabel{deletion-t-tree}
\end{figure}

In either of the preceding cases replacing $x$ with $x'$ has the effect of removing a (possibly empty) suffix from $\sigma_T(x_{T_y''}(v))$.  Therefore, the net effect of any number of these operations is to remove a (possibly empty) suffix from $\sigma_{T_y''}(v)$.  Therefore, the effect of all these operations on $\sigma_T(v)$ can be represented including one integer in the range $\{0,\ldots,h(T_y'')\}$ in $\mu_y(v)$.  This adds only $O(\log\log n)$ bits to $\mu_y(v)$.

In summary, for each $v\in S_y$, there is a bitstring $\mu_y(v)$ of length $O(k\log\log n)$ that allows us to compute $\sigma_{T_{y+1}}(v)$ from $\sigma_{T_y}(v)$.


\subsection{Adjacency Testing}

Summarizing, we have a labelling scheme for any $n$-vertex subgraph $G$ of $H\boxtimes P$ that contains the following information for each vertex $z=(v,y)\in V(G)$:

\begin{enumerate}[(PC1)]
  \item $\alpha(y)$; % of length $\log n-\log|T_y| + O(\log\log n)$;
  \item $\sigma_{T_y}(v)$; % of length $\log|T_y| + O(k+k^{-1}\log n)$;
  \item $d_{T_{y+b}}(x_{T_{y+b}}(p_j(v))$ for each $j\in\{1,\ldots,t+1\}$ and $b\in\{-1,0,1\}$; 
  \item $\varphi'(p_j(v))$ for each $j\in\{1,\ldots,t+1\}$;
  \item $\varphi'(v)$;
  \item $\mu_y(v)$;
  \item $a(z)$.
\end{enumerate}
The only one of these quantities not yet defined is $a(z)$, which is a sequence of $3t$ bits that indicate which of the potential edges joining $z$ to elements of $\{(p_j(v),y+b): j\in\{1,\ldots,t+1\},\, b\in\{-1,0,1\}\}$ are actually present in $G$.

Given the labels of $z_1:=(v_1,y_1)$ and $z_2:=(v_2,y_2)$ we test if $z_1z_2\in E(G)$ by first using $\alpha(y_1)$ and $\alpha(y_2)$ to determine which of the following applies:
\begin{enumerate}
  \item $|y_1-y_2|>1$: In this case $y_1\neq y_2$ and $y_1y_2\not\in P$, so $z_1z_2\not\in E(H\boxtimes P)$, so $z_1z_2\not\in E(G)$.

  \item $y_1=y_2$.  Let $y=y_1=y_2$.  In this case (PC2)--(PC4) contains $\tau_{y}(v_1)$ and $\tau_{y}(v_2)$ and we use this to test if $v_1v_2\in E(H)$.  If not, then $v_1v_2\not\in E(H\boxtimes P)$ so $z_1z_2\not\in E(G)$.  
  
  If $v_1v_2\in E(H)$ then we know that $z_1z_2\in E(H\boxtimes P)$.  In this case $\tau_{y}(v_1)$ and $\tau_{y}(v_2)$ also tell us that (without loss of generality) $x_1$ is the $j$-parent of $x_2$ in $H$.  We can now consult the relevant bit of $a(z_1)$ to determine if $z_1z_2\in E(G)$.

  \item $y_2-y_1=1$: Let $y=y_1$ (so that $y_2=y+1$).  We use $\mu_y(v_1)$ and $\sigma_{T_y}(v_1)$ to compute $\sigma_{T_{y+1}}(v_1)$.  Now, $\sigma_{T_{y+1}}(v_1)$ and (PC3)--(PC4) contains $\tau_{y+1}(v_1)$ and (PC2)--(PC4) contains $\tau_{y+1}(v_2)$.  We use these to test if $v_1v_2\in E(H)$.  If not, then $v_1v_2\not\in E(H\boxtimes P)$ so $z_1z_2\not\in E(G)$.  
  
  If $v_1v_2\in E(H)$ then we know that $z_1z_2\in E(H\boxtimes P)$.  In this case $\tau_{y+1}(v_1)$ and $\tau_{y+1}(v_2)$ also tell us that (without loss of generality) $v_1$ is the $j$-parent of $v_2$.  We can now consult the relevant bit of $a(z_1)$ to determine if $z_1z_2\in E(G)$.

  \item $y_2-y_1=-1$:  This case is symmetric to the preceding case, with the roles of $z_1$ and $z_2$ reversed.
\end{enumerate}

This completes the proof of our main result.

\begin{thm}\thmlabel{main}
  There exists a function $\mathds{A}:(\{0,1\}^*)^2\to\{0,1\}$ such that for every $t$-tree $H$, every path $P$, and every $n$-vertex subgraph $G$ of $H\boxtimes P$, there is a prefix-free code $\eta:V(G)\to\{0,1\}^*$ such that
  $|\eta(v)|=\log n + O(\sqrt{\log n\log\log n}+t(\log t + \log\log n))$ for every $v\in V(G)$ and, for every $v,w\in V(G)$, 
  \[  \mathds{A}(\eta(v),\eta(w)) = \begin{cases}
        1 & \text{if $vw\in E(G)$} \\
        0 & \text{if $vw\not\in E(G)$}
      \end{cases}
      \]
\end{thm}

\section{Conclusion}

Given an $n$-vertex planar graph $G$, finding an 8-tree $H$, a path $P$, and a mapping of $G$ into a subgraph of $H\boxtimes P$ can be done in $O(n^2)$ time \cite{dujmovic.joret.ea:planar}.  The process of computing the labels of $V(G)$ as described in \secref{pxp} and \secref{hxp} is has a straightforward $O(n\log n)$ time implementation.  Thus, the adjacency labels described in \thmref{main} are computable in $O(n^2)$ time for $n$-vertex planar graphs.

The adjacency testing function $\mathds{A}$ is quite simple. Even without using word parallelism, this function is straightforward to implement in $O(tk\log n)$ time.  In the case of planar graphs $t=8$ and $k=O(\sqrt{\log n\log\log n})$, so the adjacency testing procedure can be implemented in $O(\log n\sqrt{\log n\log\log n})$ time.\snote{If we want to bother, we could probably do this in $O(\log n)$ time.}

\section*{Acknowledgement}

Part of this research was conducted during the Eighth Workshop on Geometry and Graphs, held at the Bellairs Research Institute, January~31--February~7, 2020.  We are grateful to the organizers and participants for providing a stimulating research environment.

  
\bibliographystyle{plainurl}
\bibliography{labelling}

\end{document}
